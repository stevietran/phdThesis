% ************************** Thesis Abstract *****************************
% Use `abstract' as an option in the document class to print only the titlepage and the abstract.

\begin{abstract}
Mostly all human activities are affected by Atmospheric boundary layer (ABL). This is also where most air pollution phenomena are occurred. Understanding of the processes taking place in the ABL has attracted various research study.

Computational Fluid Dynamics (CFD) is increasingly being used in simulation of ABL flows. Ensuring accurate description of the ABL is an important task in any ABL flow study. This can be done by simulating the horizontally homogeneous ABL flow prior of dispersion study. Either the Reynolds Averaged Navier–Stokes (RANS) equations or Large Eddy Simulations (LES) are used for atmospheric turbulence modelling. The RANS turbulence models are still widely used in practical approach to overcome boundary conditions sensitivity and computational intensive of the LES. Two equation turbulence models: standard $k-\epsilon$ and SST $k-\omega$ turbulence models are modified substantially using open source CFD tool OpenFOAM to validate its usage in simulating ABL flow.
 
Monin-Obukhov similarity theory, well validated for flow in ABL surface layer over homogeneous surface, is used to model the profiles of velocity, turbulent kinetic energy and turbulence dissipation rate at inlet of computational domain. Consistency of these profiles across the domain are ensured by deriving the relation between turbulence model constants for horizontally homogeneous constant shear stress flow. This verified turbulence model is then validated using simulated atmospheric dispersion flow of dense gas and field experiments of Liquefied Natural Gas (LNG) vapour dispersion. For simulation of LNG vapour dispersion, the proposed model also takes into accounts the heat transfer from the ground to the vapour cloud, the effect of variable temperature on gas properties. Statistical Performance Measures (SPM) are compared with LES code FDS (Fire Dynamics Simulator) and specified dispersion code FLACS (FLame ACceleration Simulator) under Model Evaluation Protocol (MEP) proposed for LNG vapour dispersion.

\end{abstract}
