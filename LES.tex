\subsection{LES}

A means of assessing the quality of grid resolution in LES \cite{Gant2009}:
\begin{itemize}
	\item Estimations based on prior RANS results
	\item Single-grid estimators
	\item Multi-grid estimators
\end{itemize} 

\subsubsection{LES filtering}
In case instantaneous values of flow is desired and solving all scale turbulent structure by DNS is too expensive. LES filtering is employed to solve only large structure of flow, while universal small ones are still modelled. Using Farve filtering, any quantities $Q$ is decomposed to filtered quantities $\widetilde{Q}$ and unfiltered quantities $Q''$. Filtering instantaneous Navier-Stokes equations results below equations:
\begin{equation} \label{eq:LES-continuity}
\pdv{\overline{\rho}}{t}  + 
\pdv{}{x_i} \left( \rho \widetilde{u_i} \right) = 0
\end{equation}

\begin{equation} \label{eq:LES-momentum}
\pdv{}{t} \left( \overline{\rho} \widetilde{u_i} \right) + 
\pdv{}{x_j} (\overline{\rho} \widetilde{u_i} \widetilde{u_j}) =
-\pdv{\overline{p}}{x_i}
+ \pdv{\overline\tau_{ij}}{x_j}
+ \pdv{}{x_j} (-\overline\rho (\widetilde{u_i u_j} - \widetilde{u_i} \widetilde{u_j} )) 
+ \overline b_i  
\end{equation}

Farve filtered equation for  species mass fraction $Y_i$:
\begin{equation} \label{eq:LES_species}
\pdv{}{t} \left( \overline{\rho} \widetilde{Y_i} \right) + 
\pdv{}{x_j} (\overline{\rho} \widetilde{u_j} \widetilde{Y_i}) =
\pdv{\overline j_{ij}}{x_j}
+ \pdv{}{x_j} \left[-\overline\rho (\widetilde{ u_j Y_i} - \widetilde{u_j} \widetilde{Y_i}) \right] 
+ \overline\omega_i  
\end{equation}

Unclosed quantities are \cite{Poinsot2005}: subgrid (unresolved) Reynold stresses $\tau_{ij}^s = -\rho(\widetilde{u_i u_j} - \widetilde{u_i} \widetilde{u_j})$; unresolved species fluxes: $j_{j} = -\rho (\widetilde{u_j Y_i} - \widetilde{u_j} \widetilde{Y_i}$), filtered laminar diffusion fluxes $\overline\tau_{ij}$, $\overline j_{ij}$  and filtered chemical reaction rate $\overline\omega_i$.


\subsubsection{Smagorinsky model}
With eddy viscosity assumption, subgrid scale Reynold stresses $\tau_{ij}^s$ can be modelled as:
\begin{equation} \label{eq:sgs_stress}
\tau_{ij}^{s} -\frac{1}{3}\tau_{kk}^{s} \delta_{ij} =  
2 \mu_t \left( \widetilde{S_{ij}} - \frac{1}{3}\widetilde{S_{kk}} \delta_{ij} \right) 
\end{equation}

The Farve-filtered rate-of-strain tensor $\widetilde{S_{ij}}$ is defined the same as Equation~\ref{eq:rate_of_strain_tensor} with velocity field is replaced by filtered velocity:
\begin{equation} \label{eq:sgs_rate_of_strain_tensor}
\widetilde{S_{ij}} = \frac{1}{2} \left( \pdv{\widetilde{u_i}}{x_j} + \pdv{\widetilde{u_j}}{x_i} \right) 
\end{equation}
The second term in LHS is included to ensure that in case of isotropic stress tensor, its trace is equal to minus twice the kinetic energy \cite{Lilly1992}.
 
$\nu_t = \mu_t/\rho$ is \emph{subgrid scale viscosity}. It is modelled using dimensional argument of characteristic length scale and characteristic velocity scale:
\begin{equation} \label{eq:sgs_viscoscity}
\nu_t =2 C_\mu \Delta^2 (2 \widetilde{S_{ij}} \widetilde{S_{ij}})^{1/2} 
\end{equation}

With gradient-diffusion assumption, turbulent scalar flux can be written as:
\begin{equation} \label{eq:sgs_turb_scalar}
-\bar{\rho} (\widetilde{u_j \phi_i} - \widetilde{u_j} \widetilde{\phi_i}) =
\bar{\rho} \alpha_t \pdv{\phi_i}{x_j}
\end{equation}
$\alpha_t$ is turbulent diffusivity and calculated from:
\begin{equation} \label{eq:sgs_turb_diffusivity}
\bar{\rho} \alpha_t  = 
C_\alpha \bar{\rho} \Delta^2 |\tilde{S}| 
\end{equation}

\subsubsection{Dynamic approach}
Dynamic modelling concept is used to derive dimensionless scaling coefficients in subgrid scale model instead of using constant coefficients as classical models. Assume we have a term $t(u)$ is a function of field variable $u$. A filtered value of $t(u)$ can be decomposed into resolved and modelled parts: $\overline{t(u)} = t(\overline{u}) + m(\overline{u})$. We use another filter, called \emph{test filter} to examine the variance of $t(\overline{u})$ and $m(\overline{u})$. Filtered term calculated at this test filter are denoted by hat symbol: $\widehat{ \overline{t(u)}} = t(\widehat{\overline{u}}) + m(\widehat{\overline{u}})$. If we have subgrid scale identity:
\begin{equation} \label{eq:sgs_identity}
\widehat{t({\overline{u}})} - t(\widehat{\overline{u}}) =
m(\widehat{\overline{u}}) - \widehat{m({\overline{u}})}
\end{equation} 

Model for modelled part $m(\overline{u})$ Equation~\ref{eq:sgs_modelled_part}. $c$ is dimensionless coefficient, which can vary in both space and time. Substituting to Equation~\ref{eq:sgs_identity}, we have Equation~\ref{eq:sgs_identity_final}, where $c^*$ is an coefficient at test filter level. 

\begin{equation} \label{eq:sgs_modelled_part}
m(\overline{u}) = c\; s(\overline{u},\Delta)
\end{equation}  

\begin{equation} \label{eq:sgs_identity_final}
\widehat{t({\overline{u}})} - t(\widehat{\overline{u}}) =
c^*\; s(\widehat{\overline{u}},\widehat{\Delta}) - \widehat{c\; s(\overline{u},\Delta)}
\end{equation} 

Assuming $c^* = c$ and allowing $c$ pass through the test filtering operator. Leonard term $\mathcal{L} = \widehat{t({\overline{u}})} - t(\widehat{\overline{u}})$ and model term $\mathcal{M} = s(\widehat{\overline{u}}, \widehat{\Delta}) - \widehat{s(\overline{u},\Delta)}$. We have $\mathcal{L} = c \mathcal{M}$. This relation can be solve by least-square to determine single value of $c$, as Equation~\ref{eq:sgs_coefficient}.

\begin{equation} \label{eq:sgs_coefficient}
c= \frac{\langle \mathcal{L} \cdot \mathcal{M}\rangle}{\langle \mathcal{M} \cdot \mathcal{M}\rangle}
\end{equation}

Apply dynamic approach to find coefficient $C_\mu$, $C_\alpha$, where density-weighted test filtering is denoted as $\check{\overline{u}} = \widehat{\overline{\rho}\widetilde{u}}/\widehat{\overline{\rho}}$.

\begin{equation} \label{eq:sgs_turb_stress}
C_\mu= \frac{\langle \mathcal{L}_{ij} \mathcal{M}_{ij} \rangle}{2 \langle \mathcal{M}_{ij} \mathcal{M}_{ij} \rangle}
\quad
\mathcal{L}_{ij} = -
\widehat{\overline{\rho}\widetilde{u_i}\widetilde{u_j}} +
\widehat{\overline{\rho}} \check{\overline{u_i}} \check{\overline{u_j}},
\quad
\mathcal{M}_{ij} = 
\widehat{\overline{\rho}} \widehat{\Delta}^2 \check{| {\widetilde{S}} |} \check{\widetilde{S_{ij}}} - 
\widehat{\overline{\rho} {\Delta}^2 | {\widetilde{S}} | \widetilde{S_{ij}}}
\end{equation}  

\begin{equation} \label{eq:sgs_turb_diffusivity_dyn}
C_\alpha= \frac{\langle \mathcal{L}_{ij} \mathcal{M}_{ij} \rangle}{\langle \mathcal{M}_{ij} \mathcal{M}_{ij} \rangle}
\quad
\mathcal{L}_i = -
\widehat{\overline{\rho}\widetilde{u_i}\widetilde{\phi}} +
\widehat{\overline{\rho}} \check{\overline{u_i}} \check{\overline{\phi}},
\quad
\mathcal{M}_{i} = 
\widehat{\overline{\rho}} \widehat{\Delta}^2 \check{| {\widetilde{S}} |} \check{\widetilde{\pdv{\phi}{x_i}}} - 
\widehat{\overline{\rho} {\Delta}^2 | {\widetilde{S}} | \widetilde{\pdv{\phi}{x_i}}}
\end{equation}  

\begin{equation} \label{eq:sgs_scalar_var_dyn}
C_\phi= \frac{\langle \mathcal{L} \mathcal{M} \rangle}{\langle \mathcal{M} \mathcal{M} \rangle}
\quad
\mathcal{L} = 
\widehat{\overline{\rho}\widetilde{\phi}\widetilde{\phi}} -
\widehat{\overline{\rho}} \check{\widetilde{\phi}} \check{\widetilde{\phi}},
\quad
\mathcal{M} = 
\widehat{\overline{\rho}} \widehat{\Delta}^2 | \grad{\check{\widetilde{\phi}}} |^2  - 
\widehat{ \overline{\rho} {\Delta}^2 | \grad{\widetilde{\phi}} |^2 }
\end{equation} 

\subsubsection{Implementation in OpenFOAM}
The anisotropic part of turbulence shear stress $\tau_{ij}$ is approximated by relating it to the resolved rate of strain tensor $D_{ij}$
\begin{equation} 
\tau_{ij} - \frac{1}{3} \tau_{kk} \delta_{ij} \approx - 2 \nu_{sgs} \text{dev}(\overline{D}_{ij})
\end{equation}
$\text{dev}(\overline{D}_{ij})$ is deviatoric part of strain tensor $D_{ij}$.

The sub-grid scale viscosity is defined as:
\begin{equation}\label{eq:foam_nu_sgs}  
\nu_{sgs} = C_{k} \Delta \sqrt{k_{sgs}}
\end{equation}
$C_{k}$ is a model constant whose default value is $0.094$ and $\Delta$ is the filter width that defines the subgrid length scale.

The SGS kinetic energy $k_{sgs}$ is defined as:
\begin{equation} \label{eq:foam_k_sgs}
k_{sgs} = \frac{1}{2} \tau_{kk} = 
\frac{1}{2} \left( \overline{u_{k}u_{k}} - 
\overline{u}_{k}\overline{u}_{k} \right) 
\end{equation}
$k_{sgs}$ is computed with the assumption of balance between the subgrid scale energy production and dissipation Equation~\ref{eq:foam_k_sgs_equilibrium}:
\begin{equation}  \label{eq:foam_k_sgs_equilibrium} 
\overline{D} : \tau_{ij} + C_{\epsilon} \frac{k_{sgs}^{1.5}}{\Delta} = 0
\end{equation}

\subsubsection{SGS models}
There are various options of SGS models implementation in OpenFOAM
\begin{itemize}
	\item \bera{Smagorinsky}: Smagorinsky SGS model
	\item \bera{kEqn}: One equation eddy-viscosity model
	\item \bera{dynamicLagrangian}: Dynamic SGS model with Lagrangian averaging
	\item \bera{dynamicKEqn}: Dynamic one equation eddy-viscosity model
	\item \bera{WALE}: Wall-adapting local eddy-viscosity (WALE) SGS model
	\item \bera{DeardorffDiffStress}: Differential SGS Stress Equation Model
\end{itemize}

For the dynamic SGS models, the spatial averaging operations of the coefficients are often performed to stabilize the calculation. The \bera{homogeneousDynSmagorinsky} model that had been implemented in older versions takes the average of the coefficient in the whole computational domain.

\subsubsection{Calculation of filter width $\Delta$}
The method for calculating the filter width $\Delta$ is specified in the \bera{turbulenceProperties} file. Available options in OpenFOAM are as follows:
\begin{itemize}
	\item cubeRootVol
	\item maxDeltaxyz
	\item maxDeltaxyzCubeRoot
	\item smooth
	\item vanDriest
	\item Prandtl
	\item IDDESDelta
\end{itemize}

The \bera{maxDeltaxyz} option calculates  filter width of the $i$ cell $\Delta_i$ by taking the maximum distance between the cell centre $P_i$ and each face centre $F_j$:
\begin{equation}  \label{eq:deltaxyz} 
\Delta_i = \text{deltaCoeff} \times \max_{1 \le j \le n_i} \left\{ \overline{P_iF_j} \right\}
\end{equation}
\text{deltaCoeff} is the user specified constant of proportion.
