%!TEX root = ../thesis.tex
\graphicspath{{Figs/Burro/}{Figs/Falcon/}{Figs/BurroFigs/}}

\chapter{Results and discussion}

\section{Burro series}
\subsection{Velocity prediction}
\begin{figure}[htbp]
	\centering
	\begin{subfigure}[]{0.49\textwidth}
		\includegraphics[width=\textwidth]{Burro3ExpVelX140Z2}
		\caption{}
	\end{subfigure} 
	~
	\begin{subfigure}[]{0.49\textwidth}
		\includegraphics[width=\textwidth]{Burro7ExpVelX140Z2}
		\caption{}
	\end{subfigure}
	%a blank line to force the subfigure onto a new line
	
	\begin{subfigure}[]{0.49\textwidth}
		\includegraphics[width=\textwidth]{Burro8ExpVelX140Z2}
		\caption{}
	\end{subfigure} 
	~
	\begin{subfigure}[]{0.49\textwidth}
		\includegraphics[width=\textwidth]{Burro9ExpVelX140Z2}
		\caption{}
	\end{subfigure}
	\caption{Centreline point-wise concentration (a) Burro 3 (b) Burro 7 (c) Burro 8 (d) Burro 9}
	\label{fig:Point-wise velocity}
\end{figure}

\subsection{Concentration prediction}
\begin{figure}[htbp]
	\centering
	\begin{subfigure}[]{0.49\textwidth}
		\includegraphics[width=\textwidth]{Burro3ConPtX140}
		\caption{}
	\end{subfigure} 
	~
	\begin{subfigure}[]{0.49\textwidth}
		\includegraphics[width=\textwidth]{Burro7ConPtX140}
		\caption{}
	\end{subfigure}
	%a blank line to force the subfigure onto a new line
	
	\begin{subfigure}[]{0.49\textwidth}
		\includegraphics[width=\textwidth]{Burro8ConPtX57}
		\caption{}
	\end{subfigure} 
	~
	\begin{subfigure}[]{0.49\textwidth}
		\includegraphics[width=\textwidth]{Burro9ConPtX140}
		\caption{}
	\end{subfigure}
	\caption{Centreline point-wise concentration (a) Burro 3 (b) Burro 7 (c) Burro 8 (d) Burro 9}
	\label{fig:Point-wise concentration}
\end{figure}

\begin{figure}[htbp]
	\centering
	\begin{subfigure}[]{0.49\textwidth}
		\includegraphics[width=\textwidth]{Burro3SimConX140}
	\end{subfigure} 
	~
	\begin{subfigure}[]{0.49\textwidth}
		\includegraphics[width=\textwidth]{Burro3ExpConX140}
	\end{subfigure}
	%a blank line to force the subfigure onto a new line
	
	\begin{subfigure}[]{0.49\textwidth}
		\includegraphics[width=\textwidth]{Burro7SimConX140}
	\end{subfigure} 
	~
	\begin{subfigure}[]{0.49\textwidth}
		\includegraphics[width=\textwidth]{Burro7ExpConX140}
	\end{subfigure}
	%a blank line to force the subfigure onto a new line
	
	\begin{subfigure}[]{0.49\textwidth}
		\includegraphics[width=\textwidth]{Burro8SimConX140}
	\end{subfigure} 
	~
	\begin{subfigure}[]{0.49\textwidth}
		\includegraphics[width=\textwidth]{Burro8ExpConX140}
	\end{subfigure}
	%a blank line to force the subfigure onto a new line
	
	\begin{subfigure}[]{0.49\textwidth}
		\includegraphics[width=\textwidth]{Burro9SimConX140}
	\end{subfigure} 
	~
	\begin{subfigure}[]{0.49\textwidth}
		\includegraphics[width=\textwidth]{Burro9ExpConX140}
	\end{subfigure}
	\caption{Vertical isosurface at X = 140, Left: Sim, Right: Exp, Up to bottom:Burro 3, Burro 7, Burro 8 and Burro 9}
	\label{fig:IsoConX140}
\end{figure}

\begin{figure}[htbp]
	\centering
	\begin{subfigure}[]{0.49\textwidth}
		\includegraphics[width=\textwidth]{Burro3SimConZ1}
	\end{subfigure} 
	~
	\begin{subfigure}[]{0.49\textwidth}
		\includegraphics[width=\textwidth]{Burro3ExpConZ1}
	\end{subfigure}
	%a blank line to force the subfigure onto a new line
	
	\begin{subfigure}[]{0.49\textwidth}
		\includegraphics[width=\textwidth]{Burro7SimConZ1}
	\end{subfigure} 
	~
	\begin{subfigure}[]{0.49\textwidth}
		\includegraphics[width=\textwidth]{Burro7ExpConZ1}
	\end{subfigure}
	%a blank line to force the subfigure onto a new line
	
	\begin{subfigure}[]{0.49\textwidth}
		\includegraphics[width=\textwidth]{Burro8SimConZ1}
	\end{subfigure} 
	~
	\begin{subfigure}[]{0.49\textwidth}
		\includegraphics[width=\textwidth]{Burro8ExpConZ1}
	\end{subfigure}
	%a blank line to force the subfigure onto a new line
	
	\begin{subfigure}[]{0.49\textwidth}
		\includegraphics[width=\textwidth]{Burro9SimConZ1}
	\end{subfigure} 
	~
	\begin{subfigure}[]{0.49\textwidth}
		\includegraphics[width=\textwidth]{Burro9ExpConZ1}
	\end{subfigure}
	\caption{Horizontal isosurface at Z = 1 (a) Burro 3 (b) Burro 7 (c) Burro 8 (d) Burro 9}
	\label{fig:IsoConZ1}
\end{figure}

\begin{figure}[htbp]
	\centering
	\begin{subfigure}[]{0.49\textwidth}
		\includegraphics[width=\textwidth]{Burro3ConMax}
		\caption{}
	\end{subfigure} 
	~
	\begin{subfigure}[]{0.49\textwidth}
		\includegraphics[width=\textwidth]{Burro7ConMax}
		\caption{}
	\end{subfigure}
	%a blank line to force the subfigure onto a new line
	
	\begin{subfigure}[]{0.49\textwidth}
		\includegraphics[width=\textwidth]{Burro8ConMax}
		\caption{}
	\end{subfigure} 
	~
	\begin{subfigure}[]{0.49\textwidth}
		\includegraphics[width=\textwidth]{Burro9ConMax}
		\caption{}
	\end{subfigure}
	\caption{Maximum arc-wise concentration (a) Burro 3 (b) Burro 7 (c) Burro 8 (d) Burro 9}
	\label{fig:ConMax}
\end{figure}

\section{Falcon series}
\subsection{Isosurface contour}
\begin{figure}[htbp]
	\centering
	\begin{subfigure}[]{0.49\textwidth}
		\includegraphics[width=\textwidth]{Falcon1SimConX150}
	\end{subfigure} 
	%a blank line to force the subfigure onto a new line
	
	\begin{subfigure}[]{0.49\textwidth}
		\includegraphics[width=\textwidth]{Falcon3SimConX150}
	\end{subfigure} 
	%a blank line to force the subfigure onto a new line
	
	\begin{subfigure}[]{0.49\textwidth}
		\includegraphics[width=\textwidth]{Falcon4SimConX150}
	\end{subfigure} 
	%a blank line to force the subfigure onto a new line
	\caption{Vertical isosurface at X = 150, Left: Sim, Right: Exp, Up to bottom: Falcon 1, Falcon 3, Falcon 4}
	\label{fig:FalconIsoConX150}
\end{figure}

\subsection{Arcwise prediction}
\begin{figure}[htbp]
	\centering
	\begin{subfigure}[]{0.49\textwidth}
		\includegraphics[width=\textwidth]{Falcon1ConMax}
		\caption{}
	\end{subfigure} 
	~
	\begin{subfigure}[]{0.49\textwidth}
		\includegraphics[width=\textwidth]{Falcon3ConMax}
		\caption{}
	\end{subfigure}
	%a blank line to force the subfigure onto a new line
	
	\begin{subfigure}[]{0.49\textwidth}
		\includegraphics[width=\textwidth]{Falcon4ConMax}
		\caption{}
	\end{subfigure} 

	\caption{Maximum arc-wise concentration (a) Falcon 1 (b) Falcon 3 (c) Falcon 4}
	\label{fig:FalconConMax}
\end{figure}

\subsection{Point-wise prediction}
\begin{figure}[htbp]
	\centering
	\begin{subfigure}[]{0.49\textwidth}
		\includegraphics[width=\textwidth]{Falcon1ConPtX50}
		\caption{}
	\end{subfigure} 
	~
	\begin{subfigure}[]{0.49\textwidth}
		\includegraphics[width=\textwidth]{Falcon1ConPtX150}
		\caption{}
	\end{subfigure}
	%a blank line to force the subfigure onto a new line
	
	\begin{subfigure}[]{0.49\textwidth}
		\includegraphics[width=\textwidth]{Falcon3ConPtX50}
		\caption{}
	\end{subfigure} 
	~
	\begin{subfigure}[]{0.49\textwidth}
		\includegraphics[width=\textwidth]{Falcon3ConPtX150}
		\caption{}
	\end{subfigure}
	%a blank line to force the subfigure onto a new line

	\begin{subfigure}[]{0.49\textwidth}
		\includegraphics[width=\textwidth]{Falcon4ConPtX50}
		\caption{}
	\end{subfigure} 
	~
	\begin{subfigure}[]{0.49\textwidth}
		\includegraphics[width=\textwidth]{Falcon4ConPtX150}
		\caption{}
	\end{subfigure}
	\caption{Centreline point-wise concentration (a) Falcon 1 (c) Falcon 3 (e) Falcon 4}
	\label{fig:FalconPtCon}
\end{figure}

\section{Result and discussion}

\subsection{Grid sensitivity study}
Three meshes which the number of node differs in the factor of two are made for the study of mesh independence. The statistics of these meshes are in Table 8.

Meshes are used for steady state simulation which the only atmospheric inlet (the LNG vapour inlet is excluded). The wind velocity profile and turbulent profile at the inlet is modelled using Monin-Obukhov similarity theory as described in 
\begin{figure}[htbp]
	\centering
	\includegraphics[width=0.9\textwidth]{MeshIndepentStudy.png}
	\caption{The gas concentration contour from simulation (left) and experiment(right)} \label{fig:MeshIndepentStudy}
\end{figure}

\subsection{Ground heat transfer sensitivity study}
\paragraph{Distance to LFL and bifurcated structure}: Three different models of heat transfer from the ground are used to study their effect in numerical results, which summarised in Table~\ref{tab:wall-conditon}. The  distance to LFL of case 3 – the wall temperature boundary condition is closest to the validation data.
 
\begin{table}[h!]
	\caption{Wall thermal boundary condition} \label{tab:wall-conditon}
	\centering
	\begin{tabular}{lr}
		\toprule
		&  Distance to LFL (\si{\meter})\\
		\midrule
		Case 1: Adiabatic wall	 			&405.25	\\
		Case 2: Convective heat transfer 	&372.5	\\
		Case 3:	Wall temperature			&342.7\\
		\midrule		
		Validation data						&320\\
		\bottomrule
	\end{tabular}
\end{table}

The isosurfaces of \SI{5}{\percent} concentration (LFL) at \SI{1}{\meter} height, \SI{200}{\second} of all three cases are shown at Figure~\ref{fig:Isosurface-wall-heat}. Figure~\ref{fig:LFLcontourConstTempWall} also shows the bifurcated structure at cloud. The further detail results from this case are presented in the Figure~\ref{fig:ConcentZ1mT200sCompar}. The gas concentration contour is plotted side by side with the contour analysing from experiment data. The bifurcated structure is shown in the simulation but less significant than observed from experiment. Besides, It can be seen that the gas move downwind faster. The high concentration can be found very far from the source than validation data.
\begin{figure}[htbp]
	\centering
	\begin{subfigure}[]{0.7\textwidth}
		\includegraphics[width=\textwidth]{LFLcontourAdiabatic.png}
		\caption{} \label{fig:LFLcontourAdiabatic}
	\end{subfigure}
	%a blank line to force the subfigure onto a new line
	
	\begin{subfigure}[]{0.7\textwidth}
		\includegraphics[width=\textwidth]{LFLcontourConvectiveWall.png}
		\caption{} \label{fig:LFLcontourConvectiveWall}
	\end{subfigure} 
	%a blank line to force the subfigure onto a new line
	
	\begin{subfigure}[]{0.7\textwidth}
		\includegraphics[width=\textwidth]{LFLcontourConstTempWall.png}
		\caption{} \label{fig:LFLcontourConstTempWall}
	\end{subfigure}
	\caption{Isosurfaces of \SI{5}{\percent} concentration (LFL) (a) Adiabatic wall. (b) Convective heat flux. (c) Constant temperature wall.}
	\label{fig:Isosurface-wall-heat}
\end{figure} 



\paragraph{Point-wise profiles}
For further understanding the result, the point concentration and temperature data from experiment will be compared to the results. The first point is selected near the source, which is \SI{57}{\meter} downwind and the second point is \SI{140}{\meter} downwind.

Figure~\ref{fig:point-concentration} is the plot of gas concentration at \SI{1}{\meter} elevation at the of \SI{57}{\meter} downwind. It shows the good temporal trend of the simulation to the validation data. The concentration data is fairly matched excepting the peak duration (from 75 s to 100 s). The peak concentration is underestimated four times the experimental data. Figure 29 is the comparison of concentration at the point of 140m downwind. The temporal trend remains good. The concentration magnitude is really match with the validation data.is the temperature profile at 57 m downwind. This also so the gap between predictions and validation data. The difference of lowest temperature is 20 degree. The temperature profile shown in Figure 31 is better matched with validation data.

\subsection{Effect of obstacle}
An impoundment surrounding the source having the size of \SI{60x60x3}{\meter} is used to investigate effectiveness this method in reducing the distance to LFL of vapour cloud. Figure~\ref{fig:impoundment} compares isosurfaces of \SI{5}{\percent} concentration (LFL) at \SI{1}{\meter} height, at the time of \SIlist{50;100;150;200}{\second}. At \SI{200}{\second}, the distance to LFL of the case including impoundment is \SI{207}{\meter}, while the case with no impoundment is \SI{302}{\meter}. An impoundment can reduce the cloud distance to LFL \SI{40}{\percent}.
\begin{figure}[htbp]
	\centering
	\includegraphics[width=0.9\textwidth]{impoundment.png}
	\caption{Dispersion of the cloud with impoundment (left) and no impoundment(right)} \label{fig:impoundment}
\end{figure}

\section{Conclusion}
The good preliminary simulation results show the suitable of general CFD code like FLUENT for modelling physics associating in LNG spills. The atmospheric condition models including the velocity profile, turbulent kinetic energy and turbulent dissipation rate profiles by Monin-Obukhov similarity theory contributed to that successfulness.

Three ground heat transfer assumptions are simulated and compare with the validation data. The wall temperature model which assume isothermal ground (ground’s temperature is constant) gave the closest prediction of distance to LFL. This may be explained as the case had the largest heat flux from the ground to the vapour cloud so it can compensate other source of heat addition to the cloud which is not considered in the simulation.
The detail comparison with point-wise data shows that prediction at near-source point are quite conservative. The assumption of source term contribute major in this error.

The impoundment is shown to reduce the distance to LFL of the cloud to 40\%
It is shown that the preliminary results have some limitations of:

The simulation is overestimate the distance to LFL, which means the effect of ambient velocity to the cloud dispersion is overestimate and the turbulent damping effect within the cloud should pay more attention.

The concentration and temperature peak in near source distance is quite conservative in comparison with validation data. The source modelling should be more well defined for better predictions.
