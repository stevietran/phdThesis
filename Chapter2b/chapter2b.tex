%!TEX root = ../thesis.tex
\chapter{OpenFOAM Methodology}

\section{Governing equation}

\subsection{Species transport equation}

Transport equation for mass fraction $Y$ for each species $\alpha$ of a mixture, $\alpha = 1,2,...,N$ \cite{Bird2002}. 

\begin{equation} \label{eq:concentration-conservation}
\rho \left( \frac{\partial Y_\alpha}{\partial t} + \vb*{v} \cdot \vb*{\nabla}  Y_\alpha \right)
=
-\vb*{\nabla} \cdot (\vb*{j}_\alpha)
+\dot{r}
\end{equation}

Molecular mass flux $j_A$ of $A$ in binary mixture of $A$ and $B$ can be written as Fick's law:

\begin{equation}
j_A = - \rho D_{AB} \vb*{\nabla} Y_A
\end{equation}
 
The mass diffusivity $D$ is assumed to be constant for all species. This value is usually obtain from dimensionless Schmidt number $Sc$

\begin{equation} \label{Sc-number}
Sc = \frac{\mu}{\rho D}
\end{equation}

\subsection{Momentum equation}

\begin{equation}
\frac{\partial}{\partial t} (\rho \vb*{v}) 
+ \vb*{\nabla} \cdot (\rho \vb*{v} \vb*{v})
=
-\vb*{\nabla} p
-\vb*{\nabla} \cdot \mathsf{T}
+\rho \vb*{g}
\end{equation}

\begin{equation} \label{viscous-stress-tensor}
\mathsf{T} = \mu \left( \vb*{\nabla} \vb*{v} + (\vb*{\nabla} \vb*{v})^T \right)
- \frac{2}{3} \mu (\vb*{\nabla} \cdot \vb*{v}) \vb*{\delta}  
\end{equation}

\subsection{Energy equation}

\begin{equation} \label{eq:enthalphy-conservation}
\frac{\partial}{\partial t} (\rho H) 
+ \vb*{\nabla} \cdot (\rho H \vb*{v})
=
-\vb*{\nabla} \cdot \vb*{q}
-\mathsf{T}:\vb*{\nabla} \vb*{v}
+ \frac{Dp}{Dt}
+\dot{q}_k
\end{equation}

\begin{equation} \label{eq:energy-flux}
\vb*{q}= -\vb*{\nabla} T + \sum_{\alpha = 1}^{N} \frac{H_\alpha}{M_\alpha} \vb*{j_\alpha}
\end{equation}

\section{FVM}

\subsection{Transport equation discretisation}

\subsection{Spatial term}

\subsubsection{Convective term}

\begin{equation}
\div \vb*{a} = 
\frac{1}{V} \int_V \left(\div{\vb*{a}}\right) \dd{V} = 
\frac{1}{V} \oint_S \left(\vb*{n} \cdot \vb*{a} \right) \dd{S} = 
\frac{1}{V} \left(\sum_\mathrm{owner} \vb*{S} \cdot \vb*{a}_{f} 
- \sum_\mathrm{neighbor} \vb*{S} \cdot \vb*{a}_{f} \right)
\end{equation}

Flux across the cell face: $\phi = \vb*{u}_f \cdot \vb*{S}_f$
\begin{lstlisting}[caption = Divergence scheme implimentation]
divSchemes
{
default         none;
div(phi,Q)      Gauss <interpolation scheme>;
}
\end{lstlisting}

\begin{lstlisting}
fvm.lower() = -weights.internalField()*faceFlux.internalField();
fvm.upper() = fvm.lower() + faceFlux.internalField();
fvm.negSumDiag();
\end{lstlisting}

The overall weights $w$ are derived from the selected scheme and interpolation weights. The face value is calculated from this weight as:

\begin{equation}
\phi_f = \phi_N + w \left(\phi_P - \phi_N \right)
\end{equation}

\subsubsection{Diffusion term}

\begin{lstlisting}
Foam::fv::gaussGrad<Type>::gradf
(
const GeometricField<Type, fvsPatchField, surfaceMesh>& ssf,
const word& name
)
{
typedef typename outerProduct<vector, Type>::type GradType;
const fvMesh& mesh = ssf.mesh();
tmp<GeometricField<GradType, fvPatchField, volMesh>> tgGrad;
GeometricField<GradType, fvPatchField, volMesh>& gGrad=tgGrad.ref();
const labelUList& owner = mesh.owner();
const labelUList& neighbour = mesh.neighbour();
const vectorField& Sf = mesh.Sf();
Field<GradType>& igGrad = gGrad;
const Field<Type>& issf = ssf;

forAll(owner, facei)
{
GradType Sfssf = Sf[facei]*issf[facei];

igGrad[owner[facei]] += Sfssf;
igGrad[neighbour[facei]] -= Sfssf;
}
igGrad /= mesh.V();

gGrad.correctBoundaryConditions();
...
}
\end{lstlisting}

\subsubsection{Source term}

\subsection{Temporal term}

\subsection{Boundary conditions}

\subsection{Solution techniques}

\subsection{Navier-Stokes equation discretisation}

\paragraph{Momentum equation}

\begin{equation} \label{eq:foam_momentum_eqn}
\pdv{(\rho \pmb{u})}{t} + 
\grad \cdot (\rho \pmb{u} \pmb{u}) 
= 
-\grad p + \rho \pmb{g} + 
\div{2 \mu_{eff} D(\vb*{u})} -
\grad\left( \frac{2}{3} \mu_{eff} (\div{\vb*{u}}) \right)
\end{equation}

Pressure gradient and gravity force are combine in following form:

\begin{equation} \label{eq:foam_p_rgh} 
-\grad p + \rho \vb*{g} = 
-\grad p_{rgh} - \left( \vb*{g} \cdot \vb*{r} \right) \grad \rho
\end{equation}

\begin{lstlisting}{label=UEqn}
fvVectorMatrix UEqn
(
fvm::ddt(rho, U)
+ fvm::div(rhoPhi, U)
+ turbulence->divDevRhoReff(rho, U)
+ ImpTB*( - fvm::laplacian(twoPhaseProperties.muf(), U)
+ fvm::laplacian(twoPhaseProperties.mu(), U) )
- CHSP*rho*g
);
\end{lstlisting}

\paragraph{Pressure equation}
Pressure equation is used to enforce continuity constrain. Solving pressure equation can assure velocity field to satisfy continuity equation. Poison equation of pressure in Cartesian coordinate \cite{Ferziger2012}:

\begin{equation}
\pdv{}{x_i}\left(\pdv{p}{x_i}\right) = 
-\pdv{}{x_i}\left[ \pdv{}{x_j} \left(\rho u_i u_j - \tau_{ij}\right) \right] +
\pdv{\rho b_i}{x_i} + \pdv[2]{\rho}{t}
\end{equation}

Implicit method for solving momentum equation \ref{eq:foam_momentum_eqn} can be written in discretised form as: 

\begin{equation}\label{eq:discet_momentum_eqn}
A_P^{u_i} u_{i, P}^{n+1} + \sum_l A_l^{u_i} u_{i, l}^{n+1} =
Q_{u_i}^{n+1} - \left( \frac{\delta p^{n+1}}{\delta x_i}\right)_P
\end{equation}

Where $P$ is index of velocity node, $l$ denotes neighbour cell, source term $Q$ contains all explicit terms defined in the previous time step $u^n$ and other linearised terms depend on the new time step variables.

Solving Equation~\ref{eq:discet_momentum_eqn} is done by iteration method. Outer iteration counter $m$ is used to denote the current prediction $u_i^{m}$ of the actual value of $u_i^{n+1}$. Equation solved in each outer iteration is:

\begin{equation}
A_P^{u_i} u_{i, P}^{m*} + \sum_l A_l^{u_i} u_{i, l}^{m*} =
Q_{u_i}^{m-1} - \left( \frac{\delta p^{m-1}}{\delta x_i}\right)_P
\end{equation}

Velocity field can be written from above equation as:
\begin{equation} \label{eq:discete_velocity}
u_{i, P}^{m*} = 
\tilde{u}^{m*}_{i,P}-
\frac{1}{A_P^{u_i}}\left( \frac{\delta p^{m-1}}{\delta x_i}\right)_P
\end{equation}

\begin{equation} \label{eq:discrete_momentum_u_tidle}
\tilde{u}^{m*}_{i,P} = 
\frac{Q^{m-1}_{u_i} - \sum_l A_l^{u_i} u_{i, l}^{m*}}{A_P^{u_i}}
\end{equation}

Where $\tilde{u}^{m*}_{i,P}$ contains all terms excluding pressure term as presents in Equation~\ref{eq:discrete_momentum_u_tidle}. Since velocity field is calculated from previous pressure, it should be corrected to satisfy continuity equation:

\begin{equation}
\frac{(\delta u_i^{m})}{\delta x_i} = 0
\end{equation}

Discretised Poison pressure equation:

\begin{equation}\label{eq:discete_Poison}
\frac{\delta}{\delta x_i} \left[ \frac{\rho}{A_P^{u_i}} \left( \frac{\delta p^m}{\delta x_i} \right) \right]_P = 
\left[ \frac{\delta (\rho \tilde{u}^{m*}_{i})}{\delta x_i} \right]_P
\end{equation}

Pressure correction is solved instead of actual value $u_i^m = u_i^{m*} + u'$, $p^m = p^{m-1} + p'$. Equation~\ref{eq:discete_velocity}, Equation~\ref{eq:discrete_momentum_u_tidle} and Equation~\ref{eq:discete_Poison} are rewritten in term of velocity correction and pressure correction as:

\begin{equation}
u'_{i, P} = 
\tilde{u}'_{i,P}-
\frac{1}{A_P^{u_i}}\left( \frac{\delta p'}{\delta x_i}\right)_P
\end{equation}

\begin{equation}
\tilde{u}'_{i,P} = 
- \frac{\sum_l A_l^{u_i} u'_{i, l}}{A_P^{u_i}}
\end{equation}

\begin{equation} \label{eq:discete_correction_Poison}
\frac{\delta}{\delta x_i} \left[ \frac{\rho}{A_P^{u_i}} \left( \frac{\delta p'}{\delta x_i} \right) \right]_P = 
\left[ \frac{\delta (\rho {u}^{m*}_{i})}{\delta x_i} \right]_P +
\left[ \frac{\delta (\rho \tilde{u}'_{i})}{\delta x_i} \right]_P
\end{equation}

The last term in Equation~\ref{eq:discete_correction_Poison} is unknown which is neglected when solving for pressure field.

SIMPLE algorithm:
\begin{algorithm}[H] 
	\caption{SIMPLE} \label{alg:SIMPLE}	
	\begin{algorithmic}[1]
		\STATE {Calculation of fields at new time $t_{n+1}$ using previous solution of $u^n$ and $p^n$}
		\FOR {SIMPLE loop}
		\STATE {Solving momentum equation to obtain $u_i^{m*}$}
		\STATE {Solving pressure correction equation for $p'$}
		\STATE {Correct velocity to obtain $u_i^m$ and pressure $p^m$}	
		\ENDFOR
		\STATE {Advance to the next time step}
	\end{algorithmic}
\end{algorithm}

PISO algorithm:
\begin{algorithm}[H] 
	\caption{PISO} \label{alg:PISO}	
	\begin{algorithmic}[1]
		\STATE {Calculation of fields at new time $t_{n+1}$ using previous solution of $u^n$ and $p^n$}
		\STATE {Solving momentum equation to obtain $u_i^{m*}$}
		\FOR {PISO loop}
		\STATE {Solving pressure correction equation for $p'$}
		\STATE {Correct velocity to obtain $u_i^m$ and pressure $p^m$}	
		\ENDFOR
		\STATE {Advance to the next time step}
	\end{algorithmic}
\end{algorithm}

PIMPLE algorithm:
\begin{algorithm}[H] 
	\caption{PIMPLE} \label{alg:PIMPLE}	
	\begin{algorithmic}[1]
		\STATE {Calculation of fields at new time $t_{n+1}$ using previous solution of $u^n$ and $p^n$}
		\FOR {PIMPLE loop}
		\STATE {Solving momentum equation to obtain $u_i^{m*}$}
		\FOR {PIMPLE correct}
		\STATE {Solving pressure correction equation for $p'$}
		\STATE {Correct velocity to obtain $u_i^m$ and pressure $p^m$}
		\ENDFOR
		\STATE {Solving for other fields}	
		\ENDFOR
		\STATE {Advance to the next time step}
	\end{algorithmic}
\end{algorithm}


\subsection{Solving in collocated variables arrangement}


\subsection{Energy equation}
We can choose either internal energy or enthalpy as the energy solution variable. $K \equiv |\vb*{u}|^2/2$ is kinetic energy per unit mass and $h$ is the enthalpy per unit mass. The effective thermal diffusivity $\alpha_{eff}$.  The expression of the laminar thermal diffusivity changes depending on the selected thermodynamics package. When solving for enthalpy $h$, the pressure-work term $dp/dt$ can be excluded by user option.

\begin{equation} \label{eq:foam_energy_H} 
\pdv{(\rho h)}{t} + \div( \rho \vb*{u} h) + 
\pdv{(\rho K )}{t} + \div( \rho \vb*{u} K) -
\pdv{p}{t} =
\div( \alpha_{eff} \nabla h ) + 
\rho \vb*{u} \cdot \vb*{g}  
\end{equation}

\begin{equation} \label{eq:foam_alphaEff} 
\alpha_{eff} = 
\frac{\rho \nu_t}{Pr_t} + \frac{\mu}{Pr} = 
\frac{\rho \nu_t}{Pr_t} + \frac{k}{c_p} 
\end{equation}

The calculation of the temperature is done iteratively using the Newton-Raphson method. If the specific heat capacity at constant pressure $c_p$ is expressed in the form of temperature polynomial function, the temperature in the $j$-th cell $T_j$ is calculated from the following equation:
\begin{equation} 
c_p(T) = \sum_{i=0}^7 c_i T^i
\end{equation}

\begin{equation} 
\int_{T_{std}}^{T_j} \left( \sum_{i=0}^7 c_i T^i \right) dT = h_j
\end{equation}

\section{Linear solvers}
\subsection{Overview}
\paragraph{Incomplete LU decomposition}

\paragraph{Conjugate gradient method}

\begin{algorithm}[H] 
	\caption{Conjugate gradient method} \label{alg:CG-method}	
	\begin{algorithmic}[1]
		\STATE {Initialize iteration index: $k=0$, initial solution: $\phi^0 = \phi_{in}$, initial residual: $\vb*{\rho}^0 = \vb*{Q} - A\phi_{in}$, initial direction: $\vb*{p}^0=\vb*{0}$, $s_0 = 10^{30}$}
		\STATE {$k=k+1$}
		\STATE {Solving: $M \vb*{z}^k = \vb*{\rho}^{k-1}$}
		\FOR {Construct new solution, residual, search direction: }
		\STATE {$s^k = \vb*{\rho^{k-1}} \cdot \vb*{z^k}$}
		\STATE {$\beta^k = s^k/s^{k-1}$}
		\STATE {$\vb*{p}^k = \beta^k \vb*{p}^{k-1}$}
		\STATE {$\alpha^k = s^k/(\vb*{p}^{k} \cdot A \vb*{p}^{k})$}
		\STATE {$\phi^k = \phi^{k-1} + \alpha^k \vb*{p}^{k}$}
		\STATE {$\rho^k = \rho^{k-1} - \alpha^k A \vb*{p}^{k}$}
		\ENDFOR
		\STATE {Repeat until desired residual is reached}	
	\end{algorithmic}
\end{algorithm}

\paragraph{Two-grid iteration method}

\begin{algorithm}[H] 
	\caption{Two-grid iteration method} \label{alg:Two-grid-method}	
	\begin{algorithmic}[1]
		\STATE {Perform iterations on the fine grid}
		\STATE {Compute residual on fine grid}
		\STATE {Restrict residual to the coarse grid}
		\STATE {Perform iterations of correction equation on the coarse grid}
		\STATE {Interpolate correction to the fine grid}
		\STATE {Update correction on the fine grid}
		\STATE {Repeat until desired residual is reached}	
	\end{algorithmic}
\end{algorithm}

\subsection{Linear solver GAMG}
Multi-grid solvers use fast solution from coarse grid to eliminate high frequency errors and use this for finer grid. \verb|faceAreaPair| is the agglomeration algorithm for coarsening the mesh. 

\begin{lstlisting}
//agglomerate(mesh, sqrt(fvmesh.magSf().primitiveField()));
agglomerate
(
mesh,
mag
(
cmptMultiply
(
fvmesh.Sf().primitiveField()
/sqrt(fvmesh.magSf().primitiveField()),
vector(1, 1.01, 1.02)
//vector::one
)
)
);
\end{lstlisting}


\section{Boundary conditions}

\subsection{Turbulence}

\begin{equation}
\epsilon_p = \frac{C_{\mu}^{0.75} k^{1.5}}{L}
\end{equation}

\subsection{Wall function}

The near-wall region can be divided into 3three parts: the viscous sub-layer, the buffer layer and the logarithmic region.
In viscous sub-layer ($y^{+} < 5$), the flow is dominated by viscous effect. Therefore, the fluid shear stress is balanced with the wall shear stress. The velocity is linearly evolved from the wall.

\begin{equation}
u^{+} = y^{+}
\end{equation} 

In logarithmic region ($30 < y^{+} < 200$), the flow is dominated by turbulence stress. Velocity is slowly increased along the distance from wall. $\kappa = 0.41$ is the von-Karman constant

\begin{equation}
u^{+} = \frac{1}{\kappa} \ln(Ey^{+})
\end{equation}   

In buffer layer, there is a similar magnitude for viscous and turbulent stress. The velocity is not well defined which can be treat by separating into two parts. One uses linear relationship in viscous sub-layer and the other uses logarithmic function.  

Implementation of wall function in OpenFOAM is from \cite{Kalitzin2005}  

\subsection{Thermal Boundary Conditions}

\paragraph{\bera{convectiveHeatTransfer}}
It calculates the heat transfer coefficients from the following empirical correlations for forced convection heat transfer:

\begin{eqnarray} \label{eq:foam_NuPlate}
\left\{ 
\begin{array}{l} 
Nu = 0.664 Re^{\frac{1}{2}} Pr^{\frac{1}{3}} \left( Re \le 5 \times 10^5 \right) \\ 
Nu = 0.037 Re^{\frac{4}{5}} Pr^{\frac{1}{3}} \left( Re \ge 5 \times 10^5 \right)
\end{array} 
\right. 
\end{eqnarray}

\paragraph{\bera{externalWallHeatFluxTemperature}}
Specified heat flux:

\begin{equation} \label{eq:foam_fixedHeatFlux}
-k \frac{T_p – T_b}{\vert \vb*{d} \vert} = q + q_r  
\end{equation}
Specify the heat transfer coefficient $h$ and the ambient temperature $T_a$

\begin{equation} \label{eq:foam_fixedHeatTransferCoeff} 
-k \frac{T_p – T_b}{\vert \vb*{d} \vert} = 
\frac{T_a – T_b}{R_{th}} + q_r
\end{equation}

\section{Thermophysical models}

\begin{lstlisting}[caption={ThermoType dictionary}]
ThermoType
{
type            heRhoThermo;
mixture         reactingMixture;
transport       sutherland;
thermo          janaf;
energy          sensibleEnthalpy;
equationOfState perfectGas;
specie          specie;
}
\end{lstlisting} 

\begin{lstlisting}
template<class ThermoType>
const ThermoType& Foam::multiComponentMixture<ThermoType>::cellMixture
(
const label celli
) const
{
mixture_ = Y_[0][celli]/speciesData_[0].W()*speciesData_[0];  
for (label n=1; n<Y_.size(); n++)
{       
mixture_ += Y_[n][celli]/speciesData_[n].W()*speciesData_[n];
}
return mixture_; 
}
\end{lstlisting} 

A thermophysical model is constructed in OpenFOAM as a pressure-temperature $p-T$ system from which other properties are computed. It can be classified as single composition, mixture with fixed or variable composition. It can also classified according to compressibility $\psi = \left ( RT \right )^{-1}$ or density $\rho$ as based parameter.

Transport model relates to the calculation of the transport variables dynamic viscosity $\mu$, thermal conductivity $\kappa$ and thermal diffusivity $\alpha$ ( for energy and enthalpy equations). Sutherland's formula define transport properties as functions of temperature Equation~\ref{eq:sutherlandTransport}. $A_s$ is a Sutherland coefficient with units \si{kg/m.s.K^{0.5}} and $T_s$ is Sutherland temperature.

\begin{equation} \label{eq:sutherlandTransport}
\mu = A_s \frac{\sqrt{T}}{1 + T_s / T}
\end{equation}

The thermodynamic models are used to calculate the specific heat $c_p$ (at constant pressure) for the fluid, from which then the other properties are derived. JANAF tables based provide relation of $c_p$ as function of temperature as Equation~\ref{eq:janaf}. The function is evaluated between a lower and upper temperature limit $T_l$ and $T_h$. Two sets of coefficients are required. The first set is to define function for temperature from $T_c$ to $T_h$ and the second for temperatures from $T_l$ to $T_c$.

\begin{equation} \label{eq:janaf}
c_p = R ((((a_4 T + a_3) T + a_2 )T + a_1)T + a_0)
\end{equation}

Perfect gas equation of state is used ro relate density $\rho$ of a fluid and the fluid pressure and temperature $\rho = P/(RT)$.

\section{Combustion model}
\begin{lstlisting}
void YSLFModel<CombThermoType>::correct()
{
// limit the scalar dissipation rate
scalar chiLimiter = solver_.maxChi();
const scalarField& ZCells = Z_.internalField();
const scalarField& varZCells = varZ_.internalField();
const scalarField& chiCells = Chi_.internalField();
scalarField& heCells = he_.internalField();
//- Update the species and enthalpy field
if(this->active())
{
scalarList x(3, 0.0);
double Zeta;
// Interpolate for internal Field
forAll(Y_, i)
{
scalarField& YCells = Y_[i].internalField();
forAll(ZCells, cellI)
{
// Enthalpy field
if (i == 0)
{
Zeta = sqrt(varZCells[cellI]/max(ZCells[cellI]*(1 - ZCells[cellI]), SMALL));
if (useScalarDissipation_)   x[0] = min(chiCells[cellI], chiLimiter);
if (useMixtureFractionVariance_) x[1] = min(Zeta, 0.99);
x[2] = ZCells[cellI];
// Postion in the table
ubIF_[cellI] = solver_.upperBounds(x);
posIF_[cellI] = solver_.position(ubIF_[cellI], x);
// Interpolating
heCells[cellI] = solver_.interpolate(ubIF_[cellI], posIF_[cellI], (solver_.sizeTableNames() - 1));
}
// Species fields
YCells[cellI] = solver_.interpolate(ubIF_[cellI], posIF_[cellI], i);
}
}
// Interpolate for patches
forAll(he_.boundaryField(), patchi)   
{
...
}
// Calculate thermodynamic Properties
this->thermo().correct();
}
}
\end{lstlisting}

\section{RAS}

\begin{lstlisting}
kEpsilonCoeffs
{
Sc              1;
Sct             0.7;
Cchi            2;
Cmu             0.09;
C1              1.44;
C2              1.92;
C3              -0.33;
sigmak          1;
sigmaEps        1.3;
Prt             1;
}
\end{lstlisting}

\begin{lstlisting}
void kEpsilon::correct()
{
if (!turbulence_)
{
// Re-calculate viscosity
mut_ = rho_*Cmu_*sqr(k_)/epsilon_;
mut_.correctBoundaryConditions();
// Re-calculate thermal diffusivity
alphat_ = mut_/Prt_;
alphat_.correctBoundaryConditions();
return;
}
RASModel::correct();
volScalarField divU(fvc::div(phi_/fvc::interpolate(rho_)));

if (mesh_.moving())
{
divU += fvc::div(mesh_.phi());
}
tmp<volTensorField> tgradU = fvc::grad(U_);
volScalarField G(GName(), mut_*(tgradU() && dev(twoSymm(tgradU()))));
tgradU.clear();
// Update epsilon and G at the wall
epsilon_.boundaryField().updateCoeffs();
// Dissipation equation
tmp<fvScalarMatrix> epsEqn
(
fvm::ddt(rho_, epsilon_)
+ fvm::div(phi_, epsilon_)
- fvm::laplacian(DepsilonEff(), epsilon_)
==
C1_*G*epsilon_/k_
- fvm::SuSp(((2.0/3.0)*C1_ + C3_)*rho_*divU, epsilon_)
- fvm::Sp(C2_*rho_*epsilon_/k_, epsilon_)
);	
epsEqn().relax();
epsEqn().boundaryManipulate(epsilon_.boundaryField());
solve(epsEqn);
bound(epsilon_, epsilonMin_);
// Turbulent kinetic energy equation
tmp<fvScalarMatrix> kEqn
(
fvm::ddt(rho_, k_)
+ fvm::div(phi_, k_)
- fvm::laplacian(DkEff(), k_)
==
G
- fvm::SuSp((2.0/3.0)*rho_*divU, k_)
- fvm::Sp(rho_*epsilon_/k_, k_)
);	
kEqn().relax();
solve(kEqn);
bound(k_, kMin_);
// Re-calculate viscosity
mut_ = rho_*Cmu_*sqr(k_)/epsilon_;
mut_.correctBoundaryConditions();	
// Re-calculate thermal diffusivity
alphat_ = mut_/Prt_;
alphat_.correctBoundaryConditions();
}
\end{lstlisting}

\subsection{RANS}
Reynolds-Averaged Navier-Stokes (RANS) equations are listed below:
\begin{equation} \label{eq:rans-continuity}
\frac{\partial}{\partial x_i} \left( \rho \bar{u_i} \right) = 0
\end{equation}

\begin{equation} \label{eq:rans-momentum}
\frac{\partial}{\partial t} \left( \rho \bar{u_i} \right) 
+ \frac{\partial}{\partial x_j} (\rho \bar{u_i} \bar{u_j})
=
-\frac{\partial \bar{p}}{\partial x_i}
+ \frac{\partial \tau_{ij}}{\partial x_j}
+ \frac{\partial}{\partial x_j} (-\rho \overline{u'_i u'_j}) 
+ \rho g_i  
\end{equation}

\begin{equation} \label{eq:rans-scalar}
\frac{\partial}{\partial t} \left(\rho \bar{\phi} \right) 
+ \frac{\partial}{\partial x_j}(\rho \bar{u_j} \bar{\phi})
=
\frac{\partial}{\partial x_j} (\Gamma \frac{\partial \bar{\phi}}{\partial x_j})
+\frac{\partial}{\partial x_j} (-\rho \overline{u'_j \phi'})
\end{equation}

The mixture density$\rho$, $C_p$, $C_{p\alpha}$ are specific heats at constant pressure of the mixture and each species.

\begin{equation} \label{eq:density}
\rho = \frac{PM_a}{RT \left[1 + \left(\dfrac{M_a}{M_v} - 1\right) q_v \right]}
\end{equation}

\begin{equation} \label{eq:specific heat}
C_p = \sum C_{p\alpha} \omega_\alpha
\end{equation}

Turbulent fluxes are modelled using eddy-viscosity hypothesis which treat turbulent fluxes similar to laminar ones:
\begin{equation} \label{eq:momentum turbulent flux}
-\rho \overline{u'_i u'_j} 
=
\mu_t \left( \frac{\partial u_i}{\partial x_j} + \frac{\partial u_j}{\partial x_i} \right) 
- \frac{2}{3} \rho k \delta_{ij}
\end{equation}

\begin{equation} \label{eq:energy turbulent flux}
-\rho \overline{u'_i {\phi}'} 
=
\Gamma_t \frac{\partial \phi}{\partial x_j} 
\end{equation}

\begin{equation} \label{eq:species turbulent flux}
-\rho \overline{u'_i \omega'_\alpha}
=
\Gamma_{t\alpha} \frac{\partial \omega_\alpha}{\partial x_j} 
\end{equation}

Turbulent diffusivity $\Gamma_t$ can be defined from turbulence Prandtl number $Pr_{t}$:
\begin{equation} \label{eq:turb-diffusivity}
\Gamma_t =  \frac{\mu_t}{Pr_{t}}
\end{equation}

\subsubsection{The $k-\epsilon$ model} 
relies on Prandtl-Kolmogorov expression as Equation~\ref{eq:turb-viscosity} \cite{Busini2016}. 
\begin{equation} \label{eq:turb-viscosity}
\mu_t=C_\mu \frac{k^2}{\epsilon}
\end{equation}

Two additional transport equations for turbulence kinetic energy (Equation~\ref{eq:turb-kinetic-energy-eq}) and turbulence dissipation rate (\ref{eq:dissipation-rate-eq}) are required:
\begin{equation} \label{eq:turb-kinetic-energy-eq}
\frac{D}{D t} (\rho k) 
=
\frac{\partial}{\partial x_i} \left[ \left( \mu + \frac{\mu_t}{\sigma_k} \right) \frac{\partial k}{\partial x_j} \right] 
+ G_k 
+ G_b
- \rho \epsilon
\end{equation} 

\begin{equation} \label{eq:dissipation-rate-eq}
\frac{D}{D t} (\rho \epsilon)
=
\frac{\partial}{\partial x_i} \left[ \left( \mu + \frac{\mu_t}{\sigma_\epsilon} \right) \frac{\partial \epsilon}{\partial x_j} \right] 
+ C_{1} \frac{\epsilon}{k} G_k 
+ C_{1} \left( 1 - C_{3} \right) \frac{\epsilon}{k} G_b
- C_{2} \rho \frac{\epsilon^2}{k}
\end{equation} 

The model constants are: $C_{1\epsilon} = 1.44$, $C_{2\epsilon} = 1.92$, $C_{3\epsilon}=-0.8$ and $C_{3\epsilon} = 2.15$ for the unstable and stable regimes respectively \cite{Chan1997}.

$Pr_{t}$ used to define turbulent diffusivity can be parametrised by \textcite{Chan1997} as:
\begin{equation} \label{turl-Pr-Chan}
Pr_{t} = \sigma_{t0} \frac{1-10R_f}{(1-R_f)^2}
\end{equation}
The constant $\sigma_{t_0}=0.9852$, and $R_f$ is the flux Richardson number determining using equation~\ref{eq:Richardson number}:
\begin{equation} \label{eq:Richardson number}
R_f = -\frac{G_b}{G_k}
\end{equation}
$G_b$ and $G_k$ are the buoyancy and source term in Equation~\ref{eq:kinetic- energy-eqn}. The kinetic energy $k$ and dissipation rate $\epsilon$ are introduced in Equation~\ref{eq:momentum eddies diffusivity} and Equation~\ref{eq:energy/species eddies diffusivity} are closed using the transport equation for $k$ and $\epsilon$

\section{LES}

A means of assessing the quality of grid resolution in LES \cite{Gant2009}:
\begin{itemize}
	\item Estimations based on prior RANS results
	\item Single-grid estimators
	\item Multi-grid estimators
\end{itemize} 

\subsection{LES filtering}

In case instantaneous values of flow is desired and solving all scale turbulent structure by DNS is too expensive. LES filtering is employed to solve only large structure of flow, while universal small ones are still modelled. Using Farve filtering, any quantities $Q$ is decomposed to filtered quantities $\widetilde{Q}$ and unfiltered quantities $Q''$. Filtering instantaneous Navier-Stokes equations results below equations:
\begin{equation} \label{eq:LES-continuity}
\pdv{\overline{\rho}}{t}  + 
\pdv{}{x_i} \left( \rho \widetilde{u_i} \right) = 0
\end{equation}

\begin{equation} \label{eq:LES-momentum}
\pdv{}{t} \left( \overline{\rho} \widetilde{u_i} \right) + 
\pdv{}{x_j} (\overline{\rho} \widetilde{u_i} \widetilde{u_j}) =
-\pdv{\overline{p}}{x_i}
+ \pdv{\overline\tau_{ij}}{x_j}
+ \pdv{}{x_j} (-\overline\rho (\widetilde{u_i u_j} - \widetilde{u_i} \widetilde{u_j} )) 
+ \overline b_i  
\end{equation}

Farve filtered equation for  species mass fraction $Y_i$:
\begin{equation} \label{eq:LES_species}
\pdv{}{t} \left( \overline{\rho} \widetilde{Y_i} \right) + 
\pdv{}{x_j} (\overline{\rho} \widetilde{u_j} \widetilde{Y_i}) =
\pdv{\overline j_{ij}}{x_j}
+ \pdv{}{x_j} \left[-\overline\rho (\widetilde{ u_j Y_i} - \widetilde{u_j} \widetilde{Y_i}) \right] 
+ \overline\omega_i  
\end{equation}

Unclosed quantities are \cite{Poinsot2005}: subgrid (unresolved) Reynold stresses $\tau_{ij}^s = -\rho(\widetilde{u_i u_j} - \widetilde{u_i} \widetilde{u_j})$; unresolved species fluxes: $j_{j} = -\rho (\widetilde{u_j Y_i} - \widetilde{u_j} \widetilde{Y_i}$), filtered laminar diffusion fluxes $\overline\tau_{ij}$, $\overline j_{ij}$  and filtered chemical reaction rate $\overline\omega_i$.

\subsection{SGS modelling}

\subsubsection{Smagorinsky model}
With eddy viscosity assumption, subgrid scale Reynold stresses $\tau_{ij}^s$ can be modelled as:
\begin{equation} \label{eq:sgs_stress}
\tau_{ij}^{s} -\frac{1}{3}\tau_{kk}^{s} \delta_{ij} =  
2 \mu_t \left( \widetilde{S_{ij}} - \frac{1}{3}\widetilde{S_{kk}} \delta_{ij} \right) 
\end{equation}

The Farve-filtered rate-of-strain tensor $\widetilde{S_{ij}}$ is defined the same as Equation~\ref{eq:rate_of_strain_tensor} with velocity field is replaced by filtered velocity:
\begin{equation} \label{eq:sgs_rate_of_strain_tensor}
\widetilde{S_{ij}} = \frac{1}{2} \left( \pdv{\widetilde{u_i}}{x_j} + \pdv{\widetilde{u_j}}{x_i} \right) 
\end{equation}
The second term in LHS is included to ensure that in case of isotropic stress tensor, its trace is equal to minus twice the kinetic energy \cite{Lilly1992}.
 
$\nu_t = \mu_t/\rho$ is \emph{subgrid scale viscosity}. It is modelled using dimensional argument of characteristic length scale and characteristic velocity scale:
\begin{equation} \label{eq:sgs_viscoscity}
\nu_t =2 C_\mu \Delta^2 (2 \widetilde{S_{ij}} \widetilde{S_{ij}})^{1/2} 
\end{equation}

With gradient-diffusion assumption, turbulent scalar flux can be written as:
\begin{equation} \label{eq:sgs_turb_scalar}
-\bar{\rho} (\widetilde{u_j \phi_i} - \widetilde{u_j} \widetilde{\phi_i}) =
\bar{\rho} \alpha_t \pdv{\phi_i}{x_j}
\end{equation}
$\alpha_t$ is turbulent diffusivity and calculated from:
\begin{equation} \label{eq:sgs_turb_diffusivity}
\bar{\rho} \alpha_t  = 
C_\alpha \bar{\rho} \Delta^2 |\tilde{S}| 
\end{equation}

\subsubsection{Dynamic approach}
Dynamic modelling concept is used to derive dimensionless scaling coefficients in subgrid scale model instead of using constant coefficients as classical models. Assume we have a term $t(u)$ is a function of field variable $u$. A filtered value of $t(u)$ can be decomposed into resolved and modelled parts: $\overline{t(u)} = t(\overline{u}) + m(\overline{u})$. We use another filter, called \emph{test filter} to examine the variance of $t(\overline{u})$ and $m(\overline{u})$. Filtered term calculated at this test filter are denoted by hat symbol: $\widehat{ \overline{t(u)}} = t(\widehat{\overline{u}}) + m(\widehat{\overline{u}})$. If we have subgrid scale identity:
\begin{equation} \label{eq:sgs_identity}
\widehat{t({\overline{u}})} - t(\widehat{\overline{u}}) =
m(\widehat{\overline{u}}) - \widehat{m({\overline{u}})}
\end{equation} 

Model for modelled part $m(\overline{u})$ Equation~\ref{eq:sgs_modelled_part}. $c$ is dimensionless coefficient, which can vary in both space and time. Substituting to Equation~\ref{eq:sgs_identity}, we have Equation~\ref{eq:sgs_identity_final}, where $c^*$ is an coefficient at test filter level. 

\begin{equation} \label{eq:sgs_modelled_part}
m(\overline{u}) = c\; s(\overline{u},\Delta)
\end{equation}  

\begin{equation} \label{eq:sgs_identity_final}
\widehat{t({\overline{u}})} - t(\widehat{\overline{u}}) =
c^*\; s(\widehat{\overline{u}},\widehat{\Delta}) - \widehat{c\; s(\overline{u},\Delta)}
\end{equation} 

Assuming $c^* = c$ and allowing $c$ pass through the test filtering operator. Leonard term $\mathcal{L} = \widehat{t({\overline{u}})} - t(\widehat{\overline{u}})$ and model term $\mathcal{M} = s(\widehat{\overline{u}}, \widehat{\Delta}) - \widehat{s(\overline{u},\Delta)}$. We have $\mathcal{L} = c \mathcal{M}$. This relation can be solve by least-square to determine single value of $c$, as Equation~\ref{eq:sgs_coefficient}.

\begin{equation} \label{eq:sgs_coefficient}
c= \frac{\langle \mathcal{L} \cdot \mathcal{M}\rangle}{\langle \mathcal{M} \cdot \mathcal{M}\rangle}
\end{equation}

Apply dynamic approach to find coefficient $C_\mu$, $C_\alpha$, where density-weighted test filtering is denoted as $\check{\overline{u}} = \widehat{\overline{\rho}\widetilde{u}}/\widehat{\overline{\rho}}$.

\begin{equation} \label{eq:sgs_turb_stress}
C_\mu= \frac{\langle \mathcal{L}_{ij} \mathcal{M}_{ij} \rangle}{2 \langle \mathcal{M}_{ij} \mathcal{M}_{ij} \rangle}
\quad
\mathcal{L}_{ij} = -
\widehat{\overline{\rho}\widetilde{u_i}\widetilde{u_j}} +
\widehat{\overline{\rho}} \check{\overline{u_i}} \check{\overline{u_j}},
\quad
\mathcal{M}_{ij} = 
\widehat{\overline{\rho}} \widehat{\Delta}^2 \check{| {\widetilde{S}} |} \check{\widetilde{S_{ij}}} - 
\widehat{\overline{\rho} {\Delta}^2 | {\widetilde{S}} | \widetilde{S_{ij}}}
\end{equation}  

\begin{equation} \label{eq:sgs_turb_diffusivity_dyn}
C_\alpha= \frac{\langle \mathcal{L}_{ij} \mathcal{M}_{ij} \rangle}{\langle \mathcal{M}_{ij} \mathcal{M}_{ij} \rangle}
\quad
\mathcal{L}_i = -
\widehat{\overline{\rho}\widetilde{u_i}\widetilde{\phi}} +
\widehat{\overline{\rho}} \check{\overline{u_i}} \check{\overline{\phi}},
\quad
\mathcal{M}_{i} = 
\widehat{\overline{\rho}} \widehat{\Delta}^2 \check{| {\widetilde{S}} |} \check{\widetilde{\pdv{\phi}{x_i}}} - 
\widehat{\overline{\rho} {\Delta}^2 | {\widetilde{S}} | \widetilde{\pdv{\phi}{x_i}}}
\end{equation}  

\begin{equation} \label{eq:sgs_scalar_var_dyn}
C_\phi= \frac{\langle \mathcal{L} \mathcal{M} \rangle}{\langle \mathcal{M} \mathcal{M} \rangle}
\quad
\mathcal{L} = 
\widehat{\overline{\rho}\widetilde{\phi}\widetilde{\phi}} -
\widehat{\overline{\rho}} \check{\widetilde{\phi}} \check{\widetilde{\phi}},
\quad
\mathcal{M} = 
\widehat{\overline{\rho}} \widehat{\Delta}^2 | \grad{\check{\widetilde{\phi}}} |^2  - 
\widehat{ \overline{\rho} {\Delta}^2 | \grad{\widetilde{\phi}} |^2 }
\end{equation} 

\subsection{Implementation in OpenFOAM}
The anisotropic part of turbulence shear stress $\tau_{ij}$ is approximated by relating it to the resolved rate of strain tensor $D_{ij}$
\begin{equation} 
\tau_{ij} - \frac{1}{3} \tau_{kk} \delta_{ij} \approx - 2 \nu_{sgs} \text{dev}(\overline{D}_{ij})
\end{equation}
$\text{dev}(\overline{D}_{ij})$ is deviatoric part of strain tensor $D_{ij}$.

The sub-grid scale viscosity is defined as:
\begin{equation}\label{eq:foam_nu_sgs}  
\nu_{sgs} = C_{k} \Delta \sqrt{k_{sgs}}
\end{equation}
$C_{k}$ is a model constant whose default value is $0.094$ and $\Delta$ is the filter width that defines the subgrid length scale.

The SGS kinetic energy $k_{sgs}$ is defined as:
\begin{equation} \label{eq:foam_k_sgs}
k_{sgs} = \frac{1}{2} \tau_{kk} = 
\frac{1}{2} \left( \overline{u_{k}u_{k}} - 
\overline{u}_{k}\overline{u}_{k} \right) 
\end{equation}
$k_{sgs}$ is computed with the assumption of balance between the subgrid scale energy production and dissipation Equation~\ref{eq:foam_k_sgs_equilibrium}:
\begin{equation}  \label{eq:foam_k_sgs_equilibrium} 
\overline{D} : \tau_{ij} + C_{\epsilon} \frac{k_{sgs}^{1.5}}{\Delta} = 0
\end{equation}

\subsection{SGS models}

There are various options of SGS models implementation in OpenFOAM
\begin{itemize}
	\item \bera{Smagorinsky}: Smagorinsky SGS model
	\item \bera{kEqn}: One equation eddy-viscosity model
	\item \bera{dynamicLagrangian}: Dynamic SGS model with Lagrangian averaging
	\item \bera{dynamicKEqn}: Dynamic one equation eddy-viscosity model
	\item \bera{WALE}: Wall-adapting local eddy-viscosity (WALE) SGS model
	\item \bera{DeardorffDiffStress}: Differential SGS Stress Equation Model
\end{itemize}

For the dynamic SGS models, the spatial averaging operations of the coefficients are often performed to stabilize the calculation. The \bera{homogeneousDynSmagorinsky} model that had been implemented in older versions takes the average of the coefficient in the whole computational domain.

\subsection{Calculation of filter width $\Delta$}

The method for calculating the filter width $\Delta$ is specified in the \bera{turbulenceProperties} file. Available options in OpenFOAM are as follows:
\begin{itemize}
	\item cubeRootVol
	\item maxDeltaxyz
	\item maxDeltaxyzCubeRoot
	\item smooth
	\item vanDriest
	\item Prandtl
	\item IDDESDelta
\end{itemize}

The \bera{maxDeltaxyz} option calculates  filter width of the $i$ cell $\Delta_i$ by taking the maximum distance between the cell centre $P_i$ and each face centre $F_j$:
\begin{equation}  \label{eq:deltaxyz} 
\Delta_i = \text{deltaCoeff} \times \max_{1 \le j \le n_i} \left\{ \overline{P_iF_j} \right\}
\end{equation}
\text{deltaCoeff} is the user specified constant of proportion.
