%!TEX root = ../thesis.tex

\graphicspath{{Chapter2/figs/Vector/}{Chapter2/figs/}}

\chapter{FDS Methodology}

\section{Governing equation}

\subsection{Species transport equation}

Transport equation for mass fraction $Y$ for each species $\alpha$ of a mixture, $\alpha = 1,2,...,N$ \cite{Bird2002}. 

\begin{equation} \label{eq:concentration-conservation}
\rho \left( \frac{\partial Y_\alpha}{\partial t} + \vb*{v} \cdot \vb*{\nabla}  Y_\alpha \right)
=
-\vb*{\nabla} \cdot (\vb*{j}_\alpha)
+\dot{r}
\end{equation}

Molecular mass flux $j_A$ of $A$ in binary mixture of $A$ and $B$ can be written as Fick's law:

\begin{equation}
j_A = - \rho D_{AB} \vb*{\nabla} Y_A
\end{equation}
 
The mass diffusivity $D$ is assumed to be constant for all species. This value is usually obtain from dimensionless Schmidt number $Sc$

\begin{equation} \label{Sc-number}
Sc = \frac{\mu}{\rho D}
\end{equation}

\subsection{Momentum equation}

\begin{equation}
\frac{\partial}{\partial t} (\rho \vb*{v}) 
+ \vb*{\nabla} \cdot (\rho \vb*{v} \vb*{v})
=
-\vb*{\nabla} p
-\vb*{\nabla} \cdot \mathsf{T}
+\rho \vb*{g}
\end{equation}

\begin{equation} \label{viscous-stress-tensor}
\mathsf{T} = \mu \left( \vb*{\nabla} \vb*{v} + (\vb*{\nabla} \vb*{v})^T \right)
- \frac{2}{3} \mu (\vb*{\nabla} \cdot \vb*{v}) \vb*{\delta}  
\end{equation}

\subsection{Energy equation}

\begin{equation} \label{eq:enthalphy-conservation}
\frac{\partial}{\partial t} (\rho H) 
+ \vb*{\nabla} \cdot (\rho H \vb*{v})
=
-\vb*{\nabla} \cdot \vb*{q}
-\mathsf{T}:\vb*{\nabla} \vb*{v}
+ \frac{Dp}{Dt}
+\dot{q}_k
\end{equation}

\begin{equation} \label{eq:energy-flux}
\vb*{q}= -\vb*{\nabla} T + \sum_{\alpha = 1}^{N} \frac{H_\alpha}{M_\alpha} \vb*{j_\alpha}
\end{equation}

\subsection{The Velocity Divergence}

\subsection{Assumptions}
Low Mach number

The low Mach number equations are solved numerically by dividing the physical space where the fire is to be simulated into a large number of rectangular cells. Within each cell the gas velocity, temperature, etc., are assumed to be uniform; changing only with time. 

Generalized "lumped species" method

\section{Velocity-pressure coupling}
Because of the low Mach number assumption, the velocity divergence (the rate of volumetric expansion) plays an important role in the overall solution scheme.  In the FDS algorithm, the divergence is a surrogate for the energy equation.  The divergence is factored out of the conservative form of the sensible enthalpy equation and when the divergence constraint is satisfied (enforced by the momentum update and solution of the Poisson equation for pressure) the conservative form of the sensible enthalpy equation is satisfied by construction.

For the $m$th zone, with background pressure $p_m$, the divergence may be written as
\begin{equation} \label{eq:div_eos}
\div{\vb* u} = {D} - {P} \pdv{p_m}{t}
\end{equation}
where
\begin{equation} \label{eq:fds_P1}
P = \frac{1}{\overline{p}_m} - \frac{1}{\rho c_p T}
\end{equation}
and
\begin{align} \label{eq:fds_D1}
D &= \frac{1}{\rho c_p T}\left[ \dot{q}''' + \dot{q}_{\rm b}''' - \div \dot{\mathbf{q}}'' - \mathbf{u} \cdot\nabla (\rho h_{\rm s}) + w \rho_0 g_z\right] \notag\\[.1in]
&+ \frac{1}{\rho} \sum_\alpha \left(\frac{\overline{W}}{W_\alpha} - \frac{h_{\rm s,\alpha}}{c_p T} \right) \bigg[ \div (\rho D_\alpha \nabla Y_\alpha) - \mathbf{u} \cdot \nabla (\rho Y_\alpha) +\dot{m}_\alpha'''  \bigg] \notag \\[.1in]
&+ \frac{1}{\rho} \sum_\alpha \left(\frac{\overline{W}}{W_\alpha} - \frac{\int_{T_{\rm b}}^T c_{p,\alpha}(T') \, {\rm d}T'}{c_p T} \right) \, \dot{m}_{\rm b,\alpha}'''
\end{align}

\subsection{The Poisson Equation for Pressure}
Before the components of velocity can be advanced in time, an elliptic partial differential equation (known as a Poisson equation) must be solved for the pressure term, $H$. This equation is formed by taking the divergence of the momentum equation:

\begin{equation}\label{eq:poisson}
\laplacian{H} = -\pdv{(\div{\vb*{u}})}{t} - \div{\left( \vb*{F}_{\rm A} + \vb*{F}_{\rm B} \right)}
\end{equation}

Note that the perturbation pressure $\tilde{p}$ appears on both sides of Eq.~(\ref{eq:poisson}). The value of $\tilde{p}$ in $\vb*{F}_{\rm B}$ is taken from the last computed $H$. The pressure on the left hand side (incorporated in the variable $H$) is solved for directly. The reason for the decomposition of the pressure term is so that the linear algebraic system arising from the discretization of Eq.~(\ref{eq:poisson}) has constant coefficients (i.e., it is {\em separable}) and can be solved to machine accuracy by a fast, direct (i.e., non-iterative) method that utilizes Fast Fourier Transforms (FFT). As will be discussed below, the Poisson equation is solved multiple times, each time driving the old and new values of $\tilde{p}$ closer together.

The discretized form of the Poisson equation is
\begin{align}
\frac{H_{i+1,jk}-2H_{ijk}+H_{i-1,jk}}{\delta x^2} +
\frac{H_{i,j+1,k}-2H_{ijk}+H_{i,j-1,k}}{\delta y^2} +
\frac{H_{ij,k+1}-2H_{ijk}+H_{ij,k-1}}{\delta z^2} \notag \\ =
-\frac{F_{x,ijk} - F_{x,i-1,jk}}{\delta x}
-\frac{F_{y,ijk} - F_{y,i,j-1,k}}{\delta y}
-\frac{F_{z,ijk} - F_{z,ij,k-1}}{\delta z} - \frac{\delta}{\delta t}(\div{\vb*{u}})_{ijk}
\end{align}

This elliptic partial differential equation is solved using a direct FFT-based solver~\cite{Sweet:1} that is part of a library of routines for solving elliptic PDEs called CRAYFISHPAK\footnote{CRAYFISHPAK, a vectorized form of the elliptic equation solver FISHPAK, was originally developed at the National Center for Atmospheric Research (NCAR) in Boulder, Colorado.}. To ensure that the divergence of the fluid is consistent with the definition given in Eq.~(\ref{eqn_divfromeos}), the time derivative of the divergence is defined at the predictor step:
\begin{equation}
\frac{\delta}{\delta t}(\div \vb*u)_{ijk} \equiv
\frac{(\div \vb*u)_{ijk}^*
	- (\div \vb*u)_{ijk}^n}{\dd t}
\end{equation}

and then
\begin{equation}
\frac{\delta}{\delta t}(\div \vb*u)_{ijk} \equiv
\frac{(\div \vb*u)_{ijk}^{n+1} -
	 \left[ (\div \vb*u)_{ijk}^*
	+ (\div \vb*u)_{ijk}^n \right]}{\dd t/2}
\end{equation}

at the corrector step. By construction, the thermodynamic divergence defined in Eq.~(\ref{eq:div_eos}) is identically equal to the divergence defined by
\begin{equation}
(\div \vb*u)_{ijk} = \frac{u_{ijk}-u_{i-1,jk}}{\delta x} + \frac{v_{ijk}-v_{i,j-1,k}}{\delta y} + \frac{w_{ijk}-w_{ij,k-1}}{\delta z}
\end{equation}

The equivalence of the two definitions of the divergence is a result of the form of the discretized equations, the time-stepping scheme, and the direct solution of the Poisson equation for the pressure.

\subsubsection*{Pressure equation boundary conditions}

\section{Solution stability} \label{sec:stability}
In explicit schemes, stability criteria may often be understood in terms of using the time step to maintain physically realizable conditions.  Below we examine the necessary conditions for stability in the presence of advection, diffusion, and expansion of the velocity and scalar fields.

\subsection{The Courant-Friedrichs-Lewy (CFL) Constraint}
The well-known CFL constraint given by
\begin{equation}
\mbox{CFL} = \delta t \frac{\norm{\vb*{u}}}{\delta x} \approx 1
\end{equation}

places a restriction on the time step due to the advection velocity. Physically, the constraint says that a fluid element should not traverse more than one cell within a time step. For LES, this constraint has the added advantage of keeping the implicit temporal and spatial filters consistent with each other.  In other words, in order to resolve an eddy of size $\delta x$, the time step needs to be in concert with the CFL.  If one were to employ an implicit scheme for the purpose of taking time steps say 10 times larger than the CFL limit, the smallest resolvable turbulent motions would then be roughly 10 times the grid spacing, which would severely limit the benefit of LES.  In most cases, if one wishes the simulation to run faster, a better strategy is to coarsen the grid resolution while keeping the CFL $\approx 1$.

The exact CFL needed to maintain stability depends on the order (as well as other properties) of the time integration scheme and the choice of velocity norm. Three choices for velocity norm are available in FDS (set on \bera{MISC}):

\bera{CFL\_VELOCITY\_NORM=0} (default, least restrictive, corresponds to $L_\infty$ norm of velocity vector)
\begin{equation}
\frac{\|\mathbf{u}\|}{\delta x} = \max \left(\frac{|u|}{\delta x}, \frac{|v|}{\delta y}, \frac{|w|}{\delta z}\right)
\end{equation}

\bera{CFL\_VELOCITY\_NORM=1} (most restrictive, corresponds to $L_1$ norm of velocity vector)
\begin{equation}
\frac{\|\mathbf{u}\|}{\delta x} = \frac{|u|}{\delta x}+\frac{|v|}{\delta y}+\frac{|w|}{\delta z}
\end{equation}

\bera{CFL\_VELOCITY\_NORM=2} ($L_2$ norm of velocity vector)
\begin{equation}
\frac{\|\mathbf{u}\|}{\delta x} = \sqrt{ (u/\delta x)^2+(v/\delta y)^2+(w/\delta z)^2 }
\end{equation}

\subsection{The Von Neumann Constraint}
The Von Neumann constraint is given by
\begin{equation}
\mbox{VN} = \delta t \max \big[(\mu/\rho),D_\alpha \big] \; \sum_i \frac{1}{\delta x_i^2} < \frac{1}{2}
\end{equation}

We can understand this constraint in a couple of different ways.  First, we could consider the model for the diffusion velocity of species $\alpha$ in direction $i$, $V_{\alpha,i}Y_\alpha = -D_\alpha \partial Y_\alpha/\partial x_i$, and we would then see that VN is simply a CFL constraint due to diffusive transport.

We can also think of VN in terms of a total variation diminishing (TVD) constraint.  That is, if we have variation (curvature) in the scalar field, we do not want to create spurious wiggles that can lead to an instability by overshooting the smoothing step.  Consider the following explicit update of the heat equation for $u$ in 1D. Here subscripts indicate grid indices and $\nu$ is the diffusivity.
\begin{equation}
u_i^{n+1} = u_i^n + \frac{\delta t \, \nu}{\delta x^2} \left( u_{i-1}^n - 2u_i^n + u_{i+1}^n \right)
\end{equation}

Very simply, notice that if $\delta t \nu/\delta x^2 = 1/2$ then $u_i^{n+1} = (u_{i-1}^n + u_{i+1}^n)/2$.  If the time step is any larger we overshoot the straight line connecting neighboring cell values.  Of course, this restriction is only guaranteed to be TVD if the $u$ field is ``smooth'', else the neighboring cell values may be shifting in the opposite direction.  Unfortunately, in LES there is no such guarantee and so the VN constraint can be particularly devilish in generating instabilities. For this reason, some practitioners like to employ implicit methods for the diffusive terms.

\subsection{Realizable Mass Density Constraint}
In an explicit Euler update of the continuity equation, if the time increment is too large the computational cell may be totally drained of mass, which of course is not physical. The constraint $\rho^{n+1}>0$ therefore leads to the following restriction on the time step:
\begin{equation}
\label{eqn_dtmassrestrict}
\delta t < \frac{\rho^n}{\overline{\mathbf{u}}^n\cdot\nabla\rho^n + \rho^n \div\mathbf{u}^n}
\end{equation}

We can argue that the case we are most concerned with is when $\rho^n$ is near zero.  A reasonable approximation to (\ref{eqn_dtmassrestrict}) then becomes (time location suppressed, summation over $i$ is implied)
\begin{equation}
\label{eqn_divstability}
\delta t < \frac{\rho}{\overline{u}_i \left(\frac{\rho-0}{\delta x_i}\right) + \rho \div\mathbf{u}} = \left[ \frac{\overline{u}_i}{\delta x_i} + \div\mathbf{u} \right]^{-1}
\end{equation}

Equation (\ref{eqn_divstability}) basically adds the effect of thermal expansion to the CFL constraint and provides a reason to prefer \bera{CFL\_VELOCITY\_NORM=1} as the basis for the time step restriction.

\subsection{Realizable Fluid Volume Constraint}

Mass conservation tells us that the time rate of change of a fluid element with mass $\rho V$ does not change:
\begin{equation}
\label{eq:fluidelement}
\frac{\dd (\rho V)}{\dd t} = 0 
\end{equation}

Using continuity, Eq.~(\ref{eq:fluidelement}) rearranges to
\begin{equation}
\label{eq:dvoldt}
\div\mathbf{u} = \frac{1}{V} \frac{\dd V}{\dd t}
\end{equation}

where $V(t)$ is the time-dependent volume of the fluid element.  If $\div\mathbf{u}<0$, the fluid element is under compression.  In fire dynamics this usually occurs due to cooling (heat loss by radiation, for example).  Equation (\ref{eq:dvoldt}) highlights the physical interpretation of the velocity divergence as the rate of volumetric expansion of the fluid \emph{per unit volume}.

Equation (\ref{eq:dvoldt}) also implies a time step constraint.  Consider an explicit update of Eq.~(\ref{eq:dvoldt}) for the fluid volume:
\begin{equation}
V^{n+1} = V^n + \Delta t \, V^n (\div\mathbf{u})^{\!n} \,\mbox{.}
\end{equation}

If the fluid element is in compression (the divergence is negative), positivity of the fluid volume requires the time step to be limited by
\begin{equation}
\label{eq:volumedtrestriction}
\Delta t < -(\div\mathbf{u})^{\!-1} \,\mbox{.}
\end{equation}

Note that this is the analog of the positive mass density constraint when the divergence is positive and provides the rationale for using the absolute value of the divergence $|\div\mathbf{u}|$ in the final version of the CFL constraint shown below.

\subsection{Heat Transfer Constraint}

Note that the heat transfer coefficient, $h$, has units of W/(m$^2$\,K).  Thus, a velocity scale may be formed from $h/(\rho\, c_p)$.  Anytime we have a velocity scale to resolve, we have a CFL-type stability restriction.  Therefore, the heat transfer stability check loops over all wall cells to ensure $\delta t \le \delta x \,\rho \,c_p/h$.  This check is an option. It is not done by default.

\subsection{Adjusting the Time Step}

In the default LES mode of operation, the CFL is increased or decreased to remain between 0.8 and 1.  To be clear, the CFL constraint is now given by
\begin{equation}
\mbox{CFL} = \delta t \left( \frac{\|\mathbf{u}\|}{\delta x} + |\div\mathbf{u}| \right)
\end{equation}
In DNS mode, the time step is also adjusted to maintain VN between 0.4 and 0.5. If either the CFL or VN is too large then the new time step is set to 90\% of the allowable value.  If both CFL and VN are below their minimum values then the current time step is increased by 10\%.

\section{Solution Procedure}
FDS uses a second-order accurate finite-difference approximation to the governing equations on a series of connected rectilinear meshes. The flow variables are updated in time using an explicit second-order Runge-Kutta scheme.

This section describes how this algorithm is used to advance in time the density, species mass fractions, velocity components,background and perturbation pressure. Let $\rho^n$, $Y_\alpha^n$, $\vb*{u}^n$, $p_m^n$ and $H^n$ denote these variables at the $n$th time step.

\subsection{Predictor stage}
\begin{enumerate}
	\item Compute the "patch-average" velocity field $\bar{\vb*{u}}^n$ to force normal components of velocity to match at mesh interface boundaries. Note that this change in the velocity field creates an error in the divergence which is to be corrected when the velocities are advanced in time.
	
	\item Estimate $\rho$, $Y_\alpha$, and $p_m$ at the next time step with an explicit Euler step. For example, the density is estimated by 
	$\frac{\rho^*-\rho^n}{\dd t} + \div \rho^n \bar{\vb*u}^n = 0$
	
	\item Exchange values of density and mass fraction, $\rho^*$ and $Y_\alpha^*$, at mesh boundaries. The word ``exchange'' implies that information is to be passed from one mesh to another, if necessary via MPI (Message Passing Interface) calls.
	
	\item Apply boundary conditions for $\rho^*$ and $Y_\alpha^*$.
	
	\item Compute the divergence, $\widehat{\div \bm{u}^*}$, using the estimated thermodynamic quantities. Note that the hat symbol implies that the estimated velocity field, $\vb*u^*$, has not been computed yet. The calculation of the pressure field in the next step shall ensure that the divergence of the updated velocity field is the same as that which is computed here.
	
	\item Solve the Poisson equation for the pressure fluctuation with a direct solver on each individual mesh:
	\begin{equation}
	 \laplacian H^n = - \left[ \frac{ \widehat{\div \vb*u^*} -
		\widehat{\div \vb*u^n} - \div (\bar{\vb*u}^n - \vb*u^n) }{\dd t} \right] - \div \bar{F}^n  
	\end{equation}
	Note that the vector $\bar{F}^n = F(\rho^n,\bar{\vb*u}^n)$ is computed using patch-averaged velocities. Note also that the term $\div
	(\bar{\vb*u}^n - \vb*u^n)$ corrects the error in the divergence introduced by the averaging of velocity components at mesh interfaces.
	
	\item Estimate the velocity at the next time step
	\begin{equation} 
	\frac{\vb*u^* - \bar{\vb*u}^n}{\dd t} +  \bar{F}^n + \grad H^n = 0 
	\end{equation} 
	Note that the divergence of the estimated velocity field,
	$\div \vb*u^*$, is identically equal to the divergence, $\widehat{\div \vb*u^*}$, that was derived from the estimated thermodynamic
	quantities.
	
	\item Check the time step at this point to ensure that
	\begin{equation}  
	\dd t \; \hbox{max} \left( \frac{|u|}{\delta x},\frac{|v|}{\delta y},\frac{|w|}{\delta z} \right) < 1 \quad ; \quad
	2 \; \dd t \; \nu \; \left(\frac{1}{\delta x^2} + \frac{1}{\delta y^2} + \frac{1}{\delta z^2} \right) < 1 
	\end{equation} 
	If the time step is too large, it is reduced so that it satisfies both constraints and the procedure returns to the beginning of the time step. If
	the time step satisfies the stability criteria, the procedure continues to the corrector step. See Section~\ref{sec:stability} for more details on
	stability.
\end{enumerate}

\noindent This concludes the ``Predictor'' stage of the time step.  At this point, values of $H^n$ and the components of $\vb*u^*$ are exchanged at
mesh boundaries via MPI calls.

\subsection{Correction stage}
\begin{enumerate}
	\item Compute the "patch-average" velocity field $\bar{\vb*u}^*$.
	
	\item Apply the second part of the Runge-Kutta update to the mass variables. For example, the density is corrected
	$ \frac{\rho^{n+1} -  \left(\rho^n + \rho^* \right)}{\dd t/2} +  \div \rho^* \bar{\vb*u}^* = 0 $
	
	\item Exchange values of $\rho^n$ and $Y_\alpha^n$ at mesh boundaries.
	
	\item Apply boundary conditions for $\rho^n$ and $Y_\alpha^n$.
	
	\item Compute the divergence, $\widehat{\div \vb*u^{n+1}}$ from the corrected thermodynamic quantities.
	
	\item Compute the pressure fluctuation using estimated quantities
	\begin{equation} \label{eqn_corrector_poisson2} \nabla^2H^* = - \left[ \frac{ \widehat{\div\vb*u^{n+1}} -  \left( \widehat{\div \vb*u^*} +
		\widehat{\div \vb*u^n} \right) }{\dd t/2} \right]
	- \div \bar{\mathbf{F}}^*
	\end{equation}  
	Note that the same type of correction is made for the divergence error at mesh boundaries.
	
	\item Update the velocity via the second part of the Runge-Kutta scheme
	\begin{equation} 
	\frac{ \vb*u^{n+1} -  \left( \bar{\vb*u}^* + \bar{\vb*u}^n \right)}{\dd t/2} + \bar{\mathbf{F}}^* + \nabla H^*  = 0 
	\end{equation} 
	Note again that the divergence of the corrected velocity field is identically equal to the divergence that was computed earlier.
	
	\item At the conclusion of the time step, values of $H^*$ and the components of $\vb*u^{n+1}$ are exchanged at mesh boundaries via MPI calls.
	
\end{enumerate}




\section{Thermophysical models}

\begin{lstlisting}[caption={ThermoType dictionary}]
ThermoType
{
type            heRhoThermo;
mixture         reactingMixture;
transport       sutherland;
thermo          janaf;
energy          sensibleEnthalpy;
equationOfState perfectGas;
specie          specie;
}
\end{lstlisting} 

\begin{lstlisting}
template<class ThermoType>
const ThermoType& Foam::multiComponentMixture<ThermoType>::cellMixture
(
const label celli
) const
{
mixture_ = Y_[0][celli]/speciesData_[0].W()*speciesData_[0];  
for (label n=1; n<Y_.size(); n++)
{       
mixture_ += Y_[n][celli]/speciesData_[n].W()*speciesData_[n];
}
return mixture_; 
}
\end{lstlisting} 

A thermophysical model is constructed in OpenFOAM as a pressure-temperature $p-T$ system from which other properties are computed. It can be classified as single composition, mixture with fixed or variable composition. It can also classified according to compressibility $\psi = \left ( RT \right )^{-1}$ or density $\rho$ as based parameter.

Transport model relates to the calculation of the transport variables dynamic viscosity $\mu$, thermal conductivity $\kappa$ and thermal diffusivity $\alpha$ ( for energy and enthalpy equations). Sutherland's formula define transport properties as functions of temperature Eq.~(\ref{eq:sutherlandTransport}). $A_s$ is a Sutherland coefficient with units \si{kg/m.s.K^{0.5}} and $T_s$ is Sutherland temperature.

\begin{equation} \label{eq:sutherlandTransport}
\mu = A_s \frac{\sqrt{T}}{1 + T_s / T}
\end{equation}

The thermodynamic models are used to calculate the specific heat $c_p$ (at constant pressure) for the fluid, from which then the other properties are derived. JANAF tables based provide relation of $c_p$ as function of temperature as Eq.~(\ref{eq:janaf}). The function is evaluated between a lower and upper temperature limit $T_l$ and $T_h$. Two sets of coefficients are required. The first set is to define function for temperature from $T_c$ to $T_h$ and the second for temperatures from $T_l$ to $T_c$.

\begin{equation} \label{eq:janaf}
c_p = R ((((a_4 T + a_3) T + a_2 )T + a_1)T + a_0)
\end{equation}

Perfect gas equation of state is used ro relate density $\rho$ of a fluid and the fluid pressure and temperature $\rho = P/(RT)$.


\section{LES}

\subsection{SGS models}

\begin{itemize}
	\item \bera{Smagorinsky}: Smagorinsky SGS model
	\item \bera{kEqn}: One equation eddy-viscosity model
	\item \bera{dynamicLagrangian}: Dynamic SGS model with Lagrangian averaging
	\item \bera{dynamicKEqn}: Dynamic one equation eddy-viscosity model
	\item \bera{WALE}: Wall-adapting local eddy-viscosity (WALE) SGS model
	\item \bera{DeardorffDiffStress}: Differential SGS Stress Equation Model
\end{itemize}

For the dynamic SGS models, the spatial averaging operations of the coefficients are often performed to stabilize the calculation. The \bera{homogeneousDynSmagorinsky} model that had been implemented in older versions takes the average of the coefficient in the whole computational domain.

\paragraph{Implementation in OpenFOAM}

The anisotropic part is approximated by relating it to the resolved rate of strain tensor $D_{ij}$

\begin{equation} 
\tau_{ij} - \frac{1}{3} \tau_{kk} \delta_{ij} \approx - 2 \nu_{sgs} dev(\overline{D})_{ij}
\end{equation}

$C_{k}$ is a model constant whose default value is $0.094$ and $\Delta$ is the grid size that defines the subgrid length scale:

\begin{equation}\label{eq:foam_nu_sgs}  
\nu_{sgs} = C_{k} \Delta \sqrt{k_{sgs}}
\end{equation}

\begin{equation} \label{eq:foam_k_sgs}
k_{sgs} = \frac{1}{2} \tau_{kk} = 
\frac{1}{2} \left( \overline{u_{k}u_{k}} - 
\overline{u}_{k}\overline{u}_{k} \right) 
\end{equation}

The SGS kinetic energy $k_{sgs}$ is computed with the assumption of balance between the subgrid scale energy production and dissipation Eq.~(\ref{eq:foam_k_sgs_equilibrium}). 

\begin{equation}  \label{eq:foam_k_sgs_equilibrium} 
\overline{D} : \tau_{ij} + C_{\epsilon} \frac{k_{sgs}^{1.5}}{\Delta} = 0
\end{equation}

\subsection{Calculation of filter width in OpenFOAM}

The method for calculating the filter width $\Delta$ is specified in the \bera{turbulenceProperties} file. Available options in OpenFOAM are as follows:

\begin{itemize}
	\item cubeRootVol
	\item maxDeltaxyz
	\item maxDeltaxyzCubeRoot
	\item smooth
	\item vanDriest
	\item Prandtl
	\item IDDESDelta
\end{itemize}

\paragraph{Implementation in OpenFOAM}
The \bera{maxDeltaxyz} option calculates  filter width of the $i$ cell $\Delta_i$ by taking the maximum distance between the cell centre $P_i$ and each face centre $F_j$. $\rm deltaCoeff$ user specified constant of proportion. 

\begin{equation}  \label{eq:deltaxyz} 
\Delta_i = {\rm deltaCoeff} \times \max_{1 \le j \le n_i} \left\{ \overline{P_iF_j} \right\}
\end{equation}

A means of assessing the quality of grid resolution in LES \cite{Gant2009}:

\begin{itemize}
	\item Estimations based on prior RANS results
	\item Single-grid estimators
	\item Multi-grid estimators
\end{itemize} 

\subsection{LES filtering}

In case mean values of flow is not sufficient and solving all scale turbulent structure by DNS is too expensive. LES filtering is employed to solve only large structure of flow, while universal small ones are still modelled.   

Using Farve filtering, any quantities $Q$ is decomposed to filtered quantities $\widetilde{Q}$ and unfiltered quantities $Q''$.

Filtering instantaneous Navier-Stokes equations results below equations:

\begin{equation} \label{eq:LES-continuity}
\pdv{\overline{\rho}}{t}  + 
\pdv{}{x_i} \left( \rho \widetilde{u_i} \right) = 0
\end{equation}

\begin{equation} \label{eq:LES-momentum}
\pdv{}{t} \left( \overline{\rho} \widetilde{u_i} \right) + 
\pdv{}{x_j} (\overline{\rho} \widetilde{u_i} \widetilde{u_j}) =
-\pdv{\overline{p}}{x_i}
+ \pdv{\overline\tau_{ij}}{x_j}
+ \pdv{}{x_j} (-\overline\rho (\widetilde{u_i u_j} - \widetilde{u_i} \widetilde{u_j} )) 
+ \overline b_i  
\end{equation}

Farve filtered equation for  species mass fraction $Y_i$:

\begin{equation} \label{eq:LES_species}
\pdv{}{t} \left( \overline{\rho} \widetilde{Y_i} \right) + 
\pdv{}{x_j} (\overline{\rho} \widetilde{u_j} \widetilde{Y_i}) =
\pdv{\overline j_{ij}}{x_j}
+ \pdv{}{x_j} \left[-\overline\rho (\widetilde{ u_j Y_i} - \widetilde{u_j} \widetilde{Y_i}) \right] 
+ \overline\omega_i  
\end{equation}

Unclosed quantities are \cite{Poinsot2005}: subgrid unresolved Reynold stresses $\tau_{ij}^s = -\rho(\widetilde{u_i u_j} - \widetilde{u_i} \widetilde{u_j})$; unresolved species fluxes: $j_{j} = -\rho (\widetilde{u_j Y_i} - \widetilde{u_j} \widetilde{Y_i}$), filtered laminar diffusion fluxes $\overline\tau_{ij}$, $\overline j_{ij}$  and filtered chemical reaction rate $\overline\omega_i$.

\subsection{Dynamic approach}

Dynamic modelling concept is used to derive dimensionless scaling coefficients in subgrid scale model. Assume we have a term $t(u)$ is a function of field variable $u$. A filtered value of $t(u)$ can be decomposed into resolved and modelled parts: $\overline{t(u)} = t(\overline{u}) + m(\overline{u})$. We use another filter, called \emph{test filter} to examine the variance of $t(\overline{u})$ and $m(\overline{u})$. Filtered term at test filter are denoted by hat symbol: $\widehat{ \overline{t(u)}} = t(\widehat{\overline{u}}) + m(\widehat{\overline{u}})$. We have subgrid scale identity, all terms can be calculated from resolved field $\overline{u}$. This can be used for subgrid scale model calibration.

\begin{equation} \label{eq:sgs_identity}
\widehat{t({\overline{u}})} - t(\widehat{\overline{u}}) =
m(\widehat{\overline{u}}) - \widehat{m({\overline{u}})}
\end{equation} 

Model for modelled part $m(\overline{u})$ Eq.~(\ref{eq:sgs_modelled_part}). $c$ is dimensionless coefficient, which can vary in both space and time. Substituting to Eq.~(\ref{eq:sgs_identity}), we have Eq.~(\ref{eq:sgs_identity_final}), where $c^*$ is an coefficient at test filter level. 

\begin{equation} \label{eq:sgs_modelled_part}
m(\overline{u}) = c\; s(\overline{u},\Delta)
\end{equation}  

\begin{equation} \label{eq:sgs_identity_final}
\widehat{t({\overline{u}})} - t(\widehat{\overline{u}}) =
c^*\; s(\widehat{\overline{u}},\widehat{\Delta}) - \widehat{c\; s(\overline{u},\Delta)}
\end{equation} 

Assuming $c^* = c$ and allowing $c$ pass through the test filtering operator. Leonard term $\mathcal{L} = \widehat{t({\overline{u}})} - t(\widehat{\overline{u}})$ and model term $\mathcal{M} = s(\widehat{\overline{u}}, \widehat{\Delta}) - \widehat{s(\overline{u},\Delta)}$. We have $\mathcal{L} = c \mathcal{M}$. This relation can be solve by least-square to determine single value of $c$, as Eq.~(\ref{eq:sgs_coefficient}).

\begin{equation} \label{eq:sgs_coefficient}
c= \frac{\langle \mathcal{L} \cdot \mathcal{M}\rangle}{\langle \mathcal{M} \cdot \mathcal{M}\rangle}
\end{equation}  

\subsection{SGS modelling}

\paragraph{Smagorinsky model}

With eddy viscosity assumption, subgrid scale Reynold stresses $\tau_{ij}^s$ can be modelled as Eq.~(\ref{eq:sgs_stress}). The Farve-filtered rate-of-strain tensor $\widetilde{S_{ij}}$ is defined the same as Eq.~(\ref{eq:rate_of_strain_tensor}) with velocity field is replaced by filtered velocity (Eq.~(\ref{eq:sgs_rate_of_strain_tensor})) \cite{Poinsot2005}. The second term in LHS is included to ensure that in case of isotropic stress tensor, its trace is equal to minus twice the kinetic energy \cite{Lilly1992}.

\begin{equation} \label{eq:sgs_stress}
\tau_{ij}^{s} -\frac{1}{3}\tau_{kk}^{s} \delta_{ij} =  
2 \mu_t \left( \widetilde{S_{ij}} - \frac{1}{3}\widetilde{S_{kk}} \delta_{ij} \right) 
\end{equation}

\begin{equation} \label{eq:sgs_rate_of_strain_tensor}
\widetilde{S_{ij}} = \frac{1}{2} \left( \pdv{\widetilde{u_i}}{x_j} + \pdv{\widetilde{u_j}}{x_i} \right) 
\end{equation}

$\nu_t = \mu_t/\rho$ is \emph{subgrid scale viscosity}. It is modelled using dimensional argument of characteristic length scale and characteristic velocity scale.   

\begin{equation} \label{eq:sgs_viscoscity}
\nu_t =2 C_\mu \Delta^2 (2 \widetilde{S_{ij}} \widetilde{S_{ij}})^{1/2} 
\end{equation}

With gradient-diffusion assumption, turbulent scalar flux can be written as Eq.~(\ref{eq:sgs_turb_scalar}), $\alpha_t$ is turbulent diffusivity and written as Eq.~(\ref{eq:sgs_turb_diffusivity})

\begin{equation} \label{eq:sgs_turb_scalar}
-\bar{\rho} (\widetilde{u_j \phi_i} - \widetilde{u_j} \widetilde{\phi_i}) =
\bar{\rho} \alpha_t \pdv{\phi_i}{x_j}
\end{equation}

\begin{equation} \label{eq:sgs_turb_diffusivity}
\bar{\rho} \alpha_t  = 
C_\alpha \bar{\rho} \Delta^2 |\tilde{S}| 
\end{equation}

Variance of conserved scalar $\phi''$

\begin{equation} \label{eq:sgs_turb_scalar_var}
\bar{\rho} (\widetilde{\phi''} =
\bar{\rho} C_\phi \Delta^2 | \grad{\tilde{\phi}} |^2
\end{equation}

\paragraph{Dynamic model}
Apply dynamic approach to find coefficient $C_\mu$, $C_\alpha$, where density-weighted test filtering is denoted as $\check{\overline{u}} = \widehat{\overline{\rho}\widetilde{u}}/\widehat{\overline{\rho}}$.

\begin{equation} \label{eq:sgs_turb_stress}
C_\mu= \frac{\langle \mathcal{L}_{ij} \mathcal{M}_{ij} \rangle}{2 \langle \mathcal{M}_{ij} \mathcal{M}_{ij} \rangle}
\quad
\mathcal{L}_{ij} = -
\widehat{\overline{\rho}\widetilde{u_i}\widetilde{u_j}} +
\widehat{\overline{\rho}} \check{\overline{u_i}} \check{\overline{u_j}},
\quad
\mathcal{M}_{ij} = 
\widehat{\overline{\rho}} \widehat{\Delta}^2 \check{| {\widetilde{S}} |} \check{\widetilde{S_{ij}}} - 
\widehat{\overline{\rho} {\Delta}^2 | {\widetilde{S}} | \widetilde{S_{ij}}}
\end{equation}  

\begin{equation} \label{eq:sgs_turb_diffusivity_dyn}
C_\alpha= \frac{\langle \mathcal{L}_{ij} \mathcal{M}_{ij} \rangle}{\langle \mathcal{M}_{ij} \mathcal{M}_{ij} \rangle}
\quad
\mathcal{L}_i = -
\widehat{\overline{\rho}\widetilde{u_i}\widetilde{\phi}} +
\widehat{\overline{\rho}} \check{\overline{u_i}} \check{\overline{\phi}},
\quad
\mathcal{M}_{i} = 
\widehat{\overline{\rho}} \widehat{\Delta}^2 \check{| {\widetilde{S}} |} \check{\widetilde{\pdv{\phi}{x_i}}} - 
\widehat{\overline{\rho} {\Delta}^2 | {\widetilde{S}} | \widetilde{\pdv{\phi}{x_i}}}
\end{equation}  

\begin{equation} \label{eq:sgs_scalar_var_dyn}
C_\phi= \frac{\langle \mathcal{L} \mathcal{M} \rangle}{\langle \mathcal{M} \mathcal{M} \rangle}
\quad
\mathcal{L} = 
\widehat{\overline{\rho}\widetilde{\phi}\widetilde{\phi}} -
\widehat{\overline{\rho}} \check{\widetilde{\phi}} \check{\widetilde{\phi}},
\quad
\mathcal{M} = 
\widehat{\overline{\rho}} \widehat{\Delta}^2 | \grad{\check{\widetilde{\phi}}} |^2  - 
\widehat{ \overline{\rho} {\Delta}^2 | \grad{\widetilde{\phi}} |^2 }
\end{equation} 


