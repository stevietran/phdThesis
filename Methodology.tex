%!TEX root = ./thesis.tex
% Momentum equation
\nomtypeR{$\vb*{u}$}{Velocity vector}{\si{m/s}}
\nomtypeR{$p$}{Fluid pressure}{\si{N/m^2}}
\nomtypeR{$\vb*{g}$}{Gravitational acceleration vector}{\si{m/s}}
\nomtypeG{$\rho$}{Fluid density}{\si{kg/m^3}}
\nomtypeG{$\mu$}{Dynamic viscosity}{\si{kg/m s}}
\nomtypeG{$\nu$}{Kinematic viscosity}{\si{m^2/s}}
\nomtypeG{$\tau$, $\tau_{ij}$}{Viscous stress tensor}{\si{N/m^2}}
\nomtypeG{$\vb*{\delta}$}{Kronecker symbol}{}
\nomtypeR{$p_{rgh}$}{Pressure defined without hydrostatic pressure}{\si{N/m^2}}
% Pressure equation
\nomtypeR{$A$}{matrix of coefficients}{}
\nomtypeR{$p'$}{Numerical pressure correction}{\si{N/m^2}}
\nomtypeR{$u'$}{Numerical velocity correction}{\si{m/s}}
\nomtypeG{$\alpha_P$}{Numerical under-relaxation factor}{}
% Specie transport equation
\nomtypeS{$\alpha$}{Species index}
\nomtypeS{$t$}{Turbulence part of properties}
\nomtypeS{$eff$}{Sum of turbulence and laminar part of properties}
\nomtypeG{$\alpha$}{Thermal diffusivity}{\si{m^2/s}}
\nomtypeR{$D$}{Mass diffusivity}{\si{m^2/s}}
\nomtypeR{$c_p$}{Specific heat}{\si{J/kg K}}
\nomtypeR{$Y$}{Specie mass fraction}{}
\nomtypeR{$M$}{Specie molecular weight}{\si{kmol/kg}}
\nomtypeR{$h$}{Enthalpy per unit mass}{\si{J/kg}}
\nomtypeD{$Sc$}{Schmidt number}{${\nu}/{D}$}
% Turbulence model
\nomtypeA{RANS}{Reynolds-Averaged Navier-Stokes}
\nomtypeA{SST $k-\omega$}{Menter’s Shear Stress Transport $k-\omega$}
\nomtypeR{$k$}{Turbulence kinetic energy}{\si{m^2/s^2}}
\nomtypeR{$E$}{Smooth wall constant}{}
\nomtypeG{$\epsilon$}{Turbulence dissipation rate}{\si{m^2/s^3}}
\nomtypeG{$\omega$}{Turbulence specific dissipation rate}{\si{1/s}}
\nomtypeG{$\kappa$}{von Karman constant}{}
\nomtypeS{w, s}{Properties value at wall/surface}
\nomtypeS{P}{Properties at cell point adjacent to wall}
\nomtypeR{$u^+$}{Near wall region velocity scale}{}
\nomtypeR{$y^+$}{Near wall region length scale}{}
\nomtypeR{$k^+$}{Near wall scale of turbulence kinetic energy}{}
\nomtypeG{$\epsilon^+$}{Near wall scale of turbulence dissipation rate}{}
\nomtypeG{$\omega^+$}{Near wall scale of turbulence specific dissipation rate }{}
\nomtypeG{$\nu^+$}{Near wall scale of kinematic viscosity}{}
% Thermophysical properties

\chapter{OpenFOAM Methodology}
Physical properties of a fluid flow such as velocity, pressure, temperature are dependant variable of a mathematical model. However, different fluid flows are usually described by the same mathematical model. For example, conduction of heat and diffusivity of gas concentration are both modelled as diffusive process. The general scalar transport equation represents modelled terms to characterises different physical processes:

\begin{equation}
\frac{\partial \phi}{\partial t}
+ \vb*{\nabla} \cdot (\phi \vb*{u})
+ \vb*{\nabla} \cdot (\vb*{\nabla}\phi)
=
S_\phi
\end{equation}

where $\phi$ is general scalar, $\vb*{u}$ is the velocity vector, $D$ is diffusivity coefficient. Terms in Eq.~\ref{eq:general_scalar_eqn}, from left to right, are temporal term, convective term, diffusive term and source term.

The aim of any numerical method is obtaining an approximate solution of mathematical model. In other word, the governing equations in form of partial differential equations are discretised at points in space to get the solvable form of algebraic equations system. The generation of system of algebraic equation is implemented in Finite Volume Method (FVM) using two steps: domain discretisation and equation discretisation.

OpenFOAM was firstly introduced in literature by \textcite{Weller1998} based on FVM method. In this chapter, fluid governing equations are introduced in parallel with OpenFOAM implementation. OpenFOAM keywords for variables, solvers names and codes are styled in \bera{keywords} format. All other variables are defined in \textit{Nomenclature} section. Two basic OpenFOAM solver \bera{buoyantSimpleFoam} and \bera{rhoReactingBuoyantFoam} are modified substantially and adopted for ABL simulation and atmospheric gas dispersion application.

\section{Governing equations}
\subsection{Momentum equation}
The general momentum equation can be written as \cite{Ferziger2012}:

\begin{equation}
\frac{\partial}{\partial t} (\rho \vb*{u})
+ \vb*{\nabla} \cdot (\rho \vb*{u} \vb*{u})
=
-\vb*{\nabla} p
-\vb*{\nabla} \cdot \tau
+\rho \vb*{g}
\end{equation}

where the viscous stress tensor $\tau$ is written for Newtonian fluid as:

\begin{equation} \label{viscous-stress-tensor}
\tau = \mu \left( \vb*{\nabla} \vb*{u} + (\vb*{\nabla} \vb*{u})^T \right)
- \frac{2}{3} \mu (\vb*{\nabla} \cdot \vb*{u}) \vb*{\delta}
\end{equation}

Momentum equation solved in OpenFOAM is:

\begin{equation} \label{eq:foam_momentum_eqn}
\pdv{(\rho \pmb{u})}{t} +
\grad \cdot (\rho \pmb{u} \pmb{u})
=
-\grad p + \rho \pmb{g} +
\div(2 \mu_{eff} D(\vb*{u})) -
\grad\left( \frac{2}{3} \mu_{eff} (\div{\vb*{u}}) \right)
\end{equation}

The rate of strain tensor: $D(\vb*{u}) = \frac{1}{2}\left( \nabla \vb*{u} + \left(\nabla \vb*{u}\right)^{T} \right)$. OpenFOAM implementation takes into account the effect of turbulence, therefore, $\mu_{eff} = \mu+ \mu_t$ is the sum of molecular and turbulent viscosity. To account for buoyancy effect, pressure gradient and gravity force are combined to form $p_{rgh}$ field:

\begin{equation} \label{eq:foam_p_rgh}
-\grad p + \rho \vb*{g} =
-\grad p_{rgh} - \left( \vb*{g} \cdot \vb*{r} \right) \grad \rho
\end{equation}

where $\vb*{r}$ is the position vector. $p_{rgh}$ field can be seen as the pressure defined without hydrostatic pressure. $p_{rgh}$ field is solved instead of pressure $p$ in OpenFOAM.

Code implementation of the above momentum equation in OpenFOAM can be presented in Listing~\ref{lst:UEqn}.
\lstinputlisting[caption={Momemtum equation code},label=lst:UEqn]{codes/UEqn.H}

\subsection{Pressure equation}
Pressure equation is used to enforce continuity constrain. Solving pressure equation can assure velocity field to satisfy continuity equation. Poison equation of pressure in Cartesian coordinate \cite{Ferziger2012} has the form of:
\begin{equation}
\pdv{}{x_i}\left(\pdv{p}{x_i}\right) =
-\pdv{}{x_i}\left[ \pdv{}{x_j} \left(\rho u_i u_j - \tau_{ij}\right) \right] + \pdv[2]{\rho}{t}
\end{equation}

Implicit method for solving momentum Equation~(\ref{eq:foam_momentum_eqn}) can be written in discretised form as:
\begin{equation}\label{eq:discet_momentum_eqn}
A_P^{u_i} u_{i, P}^{n+1} + \sum_l A_l^{u_i} u_{i, l}^{n+1} =
Q_{u_i}^{n+1} - \left( \frac{\delta p^{n+1}}{\delta x_i}\right)_P
\end{equation}
where $P$ is index of velocity node $u_i$, $l$ denotes neighbour cells, source term $Q$ contains all explicit terms defined using velocity at the previous time step $u_i^n$ and other linearised terms depend on the new time step variables $n+1$. $A$ is sparse square coefficients matrix. The pressure term is written in symbolic difference form.

Solving Equation~(\ref{eq:discet_momentum_eqn}) is done by iterative method. Outer iteration counter $m$ is used to denote the current prediction $u_i^{m}$ of the actual value of $u_i^{n+1}$ at the current time step. Equation solved in each outer iteration is:
\begin{equation}
A_P^{u_i} u_{i, P}^{m*} + \sum_l A_l^{u_i} u_{i, l}^{m*} =
Q_{u_i}^{m-1} - \left( \frac{\delta p^{m-1}}{\delta x_i}\right)_P
\end{equation}
where $m*$ is introduced to denote the predicted value of $u_i^{m}$. This value is usually not satisfied the continuity equation. Velocity field can be written from the above equation as:
\begin{equation} \label{eq:discete_velocity}
u_{i, P}^{m*} =
\tilde{u}^{m*}_{i,P}-
\frac{1}{A_P^{u_i}}\left( \frac{\delta p^{m-1}}{\delta x_i}\right)_P
\end{equation}
\begin{equation} \label{eq:discrete_momentum_u_tidle}
\tilde{u}^{m*}_{i,P} =
\frac{Q^{m-1}_{u_i} - \sum_l A_l^{u_i} u_{i, l}^{m*}}{A_P^{u_i}}
\end{equation}
where $\tilde{u}^{m*}_{i,P}$ contains all terms excluding pressure term as presents in Equation~(\ref{eq:discrete_momentum_u_tidle}). Since velocity field is calculated from previous pressure, it should be corrected to satisfy continuity equation:
\begin{equation}
\frac{(\delta u_i^{m})}{\delta x_i} = 0
\end{equation}
The discretised Poison pressure equation derived from continuity equation is used to correct the velocity field:
\begin{equation}\label{eq:discete_Poison}
\frac{\delta}{\delta x_i} \left[ \frac{\rho}{A_P^{u_i}} \left( \frac{\delta p^m}{\delta x_i} \right) \right]_P =
\left[ \frac{\delta (\rho \tilde{u}^{m*}_{i})}{\delta x_i} \right]_P
\end{equation}

Solving discretised Poison equation results a velocity field satisfies the continuity equation. However, the corrected velocity field and pressure field no longer satisfy the momentum equation, i.e. Equation~\ref{eq:discet_momentum_eqn}. Therefore, other outer iterations are performed until both momentum and continuity equation are satisfied.

In practice, pressure correction $p'$ and velocity correction $u'$ are solved instead of actual values:
\begin{equation} \label{eq:vel_correction}
\begin{aligned}
u_i^m = u_i^{m*} + u' \\
p^m = p^{m-1} + p'
\end{aligned}
\end{equation}
Equation~(\ref{eq:discete_velocity}),~(\ref{eq:discrete_momentum_u_tidle}) and (\ref{eq:discete_Poison}) are rewritten in term of velocity correction and pressure correction as:
\begin{equation} \label{eq:vel_correction_eqn}
u'_{i, P} =
\tilde{u}'_{i,P}-
\frac{1}{A_P^{u_i}}\left( \frac{\delta p'}{\delta x_i}\right)_P
\end{equation}
where $\tilde{u}'_{i,P}$ is written as:
\begin{equation} \label{eq:vel_correction_tilde}
\tilde{u}'_{i,P} =
- \frac{\sum_l A_l^{u_i} u'_{i, l}}{A_P^{u_i}}
\end{equation}
The pressure correction equation:
\begin{equation} \label{eq:discete_correction_Poison}
\frac{\delta}{\delta x_i} \left[ \frac{\rho}{A_P^{u_i}} \left( \frac{\delta p'}{\delta x_i} \right) \right]_P =
\left[ \frac{\delta (\rho {u}^{m*}_{i})}{\delta x_i} \right]_P +
\left[ \frac{\delta (\rho \tilde{u}'_{i})}{\delta x_i} \right]_P
\end{equation}
The last term in pressure correction Equation~(\ref{eq:discete_correction_Poison}) is unknown which is neglected when solving for pressure field. When convergence solution is reached, the velocity correction approaches zero and so does this term. However, this results in a slow convergence rate of pressure field. Momentum-pressure coupling algorithm presented in the next section is employed to achieve a better rate of convergence.

The pressure Poisson equation \bera{pEqn} implemented in OpenFOAM is shown in Listing~\ref{lst:pEqn}.
\lstinputlisting[caption={Pressure poisson equation implementation in OpenFOAM}, label=lst:pEqn] {codes/pEqn.H}

\subsection{Momentum-pressure coupling algorithms}
\paragraph{SIMPLE}
Semi-Implicit Method for Pressure Linked Equations (SIMPLE) algorithm overcomes the slow convergence issue resulting from neglecting the term in Equation~(\ref{eq:discete_correction_Poison}) by updating velocity field using Equation~(\ref{eq:vel_correction}) and (\ref{eq:vel_correction_eqn}). To improve stability of a computation, under-relaxation is combined with SIMPLE. The variable change is limited from one iteration to the next:

\begin{equation}
p^m = p^{m-1} + \alpha_p p'
\end{equation}

The SIMPLE algorithm is listed in Algorithm~\ref{alg:SIMPLE}.

\begin{algorithm}[H]
	\caption{SIMPLE algorithm} \label{alg:SIMPLE}
	\begin{algorithmic}[1]
		\STATE {Calculation of fields at new time $t_{n+1}$ using previous solution of $u^n$ and $p^n$}
		\FOR {SIMPLE loop}
		\STATE {Solving momentum equation to obtain $u_i^{m*}$}
		\STATE {Solving pressure correction equation for $p'$}
		\STATE {Correct velocity to obtain $u_i^m$ and pressure $p^m$}
		\ENDFOR
		\STATE {Advance to the next time step}
	\end{algorithmic}
\end{algorithm}

\paragraph{PISO}
Pressure-Implicit with Splitting of Operators (PISO) algorithm uses extra correction steps called inner correctors. In this second correction step, the velocity field correction is written similar to Equation~(\ref{eq:vel_correction_eqn}):

\begin{equation}
u''_{i, P} =
\tilde{u}'_{i,P}-
\frac{1}{A_P^{u_i}}\left( \frac{\delta p''}{\delta x_i}\right)_P
\end{equation}

In this second corrector step, the $\tilde{u}'_{i,P}$ term is not neglected but determined from Equation~(\ref{eq:vel_correction_tilde}) using velocity field $u'_i$ calculated in the first correction step.

The PISO is summarised in Algorithm~\ref{alg:PISO}.
\begin{algorithm}[H]
	\caption{PISO algorithm} \label{alg:PISO}
	\begin{algorithmic}[1]
		\STATE {Calculation of fields at new time $t_{n+1}$ using previous solution of $u^n$ and $p^n$}
		\STATE {Solving momentum equation to obtain $u_i^{m*}$}
		\FOR {PISO loop}
		\STATE {Solving pressure correction equation for $p'$}
		\STATE {Correct velocity to obtain $u_i^m$ and pressure $p^m$}
		\ENDFOR
		\STATE {Advance to the next time step}
	\end{algorithmic}
\end{algorithm}

\paragraph{PIMPLE}
PIMPLE is a combined algorithm of PISO and SIMPLE. The number of outer correctors defines the outer iterations where the system of equations are solved until getting convergence solution. Pressure field is corrected within each iteration defined as inner correctors. PIMPLE can be seen as performing SIMPLE for each time step and usually used for transient simulation.

PIMPLE is presented in Algorithm~\ref{alg:PIMPLE}.

\begin{algorithm}[H]
	\caption{PIMPLE algorithm} \label{alg:PIMPLE}
	\begin{algorithmic}[1]
		\STATE {Calculation of fields at new time $t_{n+1}$ using previous solution of $u^n$ and $p^n$}
		\FOR {PIMPLE outer correctors}
		\STATE {Solving momentum equation to obtain $u_i^{m*}$}
		\FOR {PIMPLE inner correctors}
		\STATE {Solving pressure correction equation for $p'$}
		\STATE {Correct velocity to obtain $u_i^m$ and pressure $p^m$}
		\ENDFOR
		\STATE {Solving for other fields}
		\ENDFOR
		\STATE {Advance to the next time step}
	\end{algorithmic}
\end{algorithm}

\subsection{Species transport equation} \label{sec:species-transport-equation}

Transport equation for mass fraction $Y$ for each species $\alpha$ of a mixture, $\alpha = 1,2,...,N$ \cite{Bird2002}:
\begin{equation} \label{eq:concentration-conservation}
\frac{\partial \rho Y_\alpha}{\partial t} + \div \left( \rho Y_\alpha \vb*{u} \right)
=
-\div(- \rho D_{eff} \grad{Y_\alpha})
+\dot{r_\alpha}
\end{equation}
$\dot{r_\alpha}$ is the reaction rate of specie $\alpha$. In the scope of this thesis, non-reacting flow is in concerned, therefore this term is neglected. Specie mass flux $\vb*{j}_\alpha$ is written in form of Fick's law:
\begin{equation} \label{eq:species_mass_flux}
\vb*{j}_\alpha
=
-\div(- \rho D_{eff} \grad{Y_\alpha})
\end{equation}
 The effective mass diffusivity $D_{eff}$ is obtained from dimensionless Schmidt number $Sc$ and effective momentum viscosity:
\begin{equation} \label{Sc-number}
Sc = \frac{\mu_{eff}}{\rho D_{eff}}
\end{equation}

In OpenFOAM, $Sc=1$, so mass diffusivity is assumed to be the same as viscosity. The code fragments implementation of Equation~(\ref{eq:concentration-conservation}) is shown in Listing~\ref{lst:YEqn}:
\lstinputlisting[caption={Species transport equation},label=lst:YEqn]{codes/YEqn.H}

\subsection{Energy equation}
Energy conservation equation can be written for total fluid enthalpy $H$ as \cite{Bird2002}:
\begin{equation} \label{eq:enthalphy-conservation}
\frac{\partial}{\partial t} (\rho H)
+ \vb*{\nabla} \cdot (\rho H \vb*{u})
=
-\vb*{\nabla} \cdot \vb*{q}
-\mathsf{T}:\vb*{\nabla} \vb*{u}
+ \frac{Dp}{Dt}
+\dot{q}_k
\end{equation}
The energy flux $q$ take into account heat diffusion and mass diffusion as:
\begin{equation} \label{eq:energy-flux}
\vb*{q}= -\vb*{\nabla} T + \sum_{\alpha = 1}^{N} \frac{H_\alpha}{M_\alpha} \vb*{j_\alpha}
\end{equation}
where $H_\alpha$ and $M_\alpha$ are total enthalpy and molecular weight of species $\alpha$. Specie mass flux $\vb*{j_\alpha}$ is calculated as Equation~\ref{eq:species_mass_flux}.

In OpenFOAM, either internal energy or enthalpy can be used as the energy variable. The energy conservation equations for enthalpy per unit mass variable $h$, which is the sum of internal energy and kinematic pressure $h \equiv e + p/\rho$, can be written as:
\begin{equation} \label{eq:foam_energy_H}
\pdv{(\rho h)}{t} + \div( \rho \vb*{u} h) +
\pdv{(\rho K )}{t} + \div( \rho \vb*{u} K) -
\pdv{p}{t} =
\div( \alpha_{eff} \nabla h ) +
\rho \vb*{u} \cdot \vb*{g}
\end{equation}
In the above equation, $K \equiv |\vb*{u}|^2/2$ is kinetic energy per unit mass, the pressure-work term $dp/dt$ can be excluded by user option, the effective thermal diffusivity $\alpha_{eff}$ is the sum of laminar and turbulent thermal diffusivity:
\begin{equation} \label{eq:foam_alphaEff}
\alpha_{eff} =
\frac{\rho \nu_t}{Pr_t} + \frac{\mu}{Pr} =
\frac{\rho \nu_t}{Pr_t} + \frac{k}{c_p}
\end{equation}

OpenFOAM implementation of energy equation using enthalpy $h$ variable is presented in Listing~\ref{lst:EEqn}.
\lstinputlisting[caption={OpenFOAM implementation of Energy equation},label={lst:EEqn}]{codes/EEqn.H}

The calculation of the temperature is done iteratively using the Newton-Raphson method from solution of energy variable. If the specific heat capacity at constant pressure $c_p$ is expressed in the form of temperature polynomial function:
\begin{equation}
c_p(T) = \sum_{i=0}^7 c_i T^i
\end{equation}
The temperature in the $j$-th cell $T_j$ is calculated from the following equation:
\begin{equation}
\int_{T_{std}}^{T_j} \left( \sum_{i=0}^7 c_i T^i \right) dT = h_j
\end{equation}

Calculate the temperature from the sensible enthalpy $h$ can be done from \bera{thermo.correct()} after solving the energy equation (Listing \ref{lst:EEqn}).

\section{Geometry and Mesh data structure}
The mesh data is stored in \bera{constant/polyMesh} directory. The typical mesh data files include \bera{points}, \bera{faces}, \bera{owner}, \bera{neighbour} and \bera{boundary}. These files are interconnected to define the mesh.

Unlike a structure mesh constructed in the per-point basis, a mesh in OpenFOAM is constructed on a per-face basis. It characterises in the owner-neighbour addressing used to determine the topological mesh structure. In owner-neighbour addressing, the ordered list of indexes in the mesh faces file is used to define the owner cells and neighbour cells where owner cells are ones have lower index. The face area normal is directed from the owner to the neighbour cell. This can help to reduce the redundancy of computing discretised operations.

\section{Numerical schemes}
\subsection{Time discretisation term}
Discretisation schemes are defined using \bera{ddtSchemes} sub-dictionary. Some selected options are:
\begin{itemize}
	\item \bera{steadyState}: used for steady state simulation.
	\item \bera{Euler}: used for transient simulation. It is the implicit first order scheme for the time derivatives. The scheme is unconditionally bounded.
	\item \bera{CrankNicolson}: also used for transient simulation. It is the implicit, second order scheme. An off-centering coefficient is requires for the scheme to be bounded.
\end{itemize}

\subsection{Convective term}
The Gauss integration of scalar field $\phi$ convection term due to the velocity field can be written as:
\begin{equation}
\div (\vb*{u}\phi) =
\frac{1}{V} \int_V \left(\div{\vb*{u} \phi}\right) \dd{V} =
\frac{1}{V} \oint_S \left(\vb*{n} \cdot \vb*{u} \phi \right) \dd{S} =
\frac{1}{V} \left(\sum_\mathrm{owner} \vb*{S} \cdot \vb*{u}_{f} \phi
- \sum_\mathrm{neighbor} \vb*{S} \cdot \vb*{u}_{f} \phi \right)
\end{equation}

In OpenFOAM, \bera{divSchemes} sub-dictionary contains parameters for discretisation of the term $\div{\Phi Q}$, where $Q$ is the scalar field and $\Phi$ is a surface mass flux defined as:
\begin{equation}
\Phi = \vb*{u}_f \cdot \vb*{S}_f
\end{equation}

An interpolation scheme is required for the calculation of $\Phi$. $\Phi$ is calculated using:
\begin{equation}
\Phi_f = \Phi_N + w \left(\Phi_P - \Phi_N \right)
\end{equation}
where $w$ is the overall weight derived from the selected interpolation scheme. The subscript $f$ represents value at a surface. The subscript $N$ and $P$ denote two adjacent cell centre points of that surface.

An example of selection and implementation for convection term of field $Q$ is seen in Listing~\ref{lst:divScheme}.
\begin{lstlisting}[caption={Selection and divergence scheme implimentation}, label={lst:divScheme}]
divSchemes
{
	default         none;
	div(phi,Q)      Gauss <interpolation scheme>;
}

// Calculation of matrix coefficient from divergence scheme selection
fvm.lower() = -weights.internalField()*faceFlux.internalField();
fvm.upper() = fvm.lower() + faceFlux.internalField();
fvm.negSumDiag();
\end{lstlisting}



\subsection{Diffusion term}
\bera{Gauss} scheme is used for discretisation of Laplacian term $\div{(\nu \grad \vb*{u})}$, which is diffusion term in momentum equation.  Interpolation scheme for the diffusion coefficient and surface normal gradient scheme for evaluating $(\grad \vb*{u})$ are also required. All parameters are defined in \bera{laplacianSchemes} sub-dictionary.

Surface normal gradient schemes required for Laplacian term is defined at face surface. \bera{orthogonal} scheme is second order accurate and applied when the vector connecting the cell centres is orthogonal to the face. In case of non-orthogonality, \bera{corrected} scheme is used to maintain second-order accuracy, i.e. an explicit non-orthogonal correction is added. \bera{limited corrected} scheme is used for severe non-orthogonality mesh which may lead to unstable solution.

An example of Gradient scheme implementation is listed below:
\begin{lstlisting}[caption={Discretisation of diffusion term}]
Foam::fv::gaussGrad<Type>::gradf
(
	const GeometricField<Type, fvsPatchField, surfaceMesh>& ssf,
	const word& name
)
{
	typedef typename outerProduct<vector, Type>::type GradType;
	const fvMesh& mesh = ssf.mesh();
	tmp<GeometricField<GradType, fvPatchField, volMesh>> tgGrad;
	GeometricField<GradType, fvPatchField, volMesh>& gGrad=tgGrad.ref();
	const labelUList& owner = mesh.owner();
	const labelUList& neighbour = mesh.neighbour();
	const vectorField& Sf = mesh.Sf();
	Field<GradType>& igGrad = gGrad;
	const Field<Type>& issf = ssf;

	forAll(owner, facei)
	{
		GradType Sfssf = Sf[facei]*issf[facei];
		igGrad[owner[facei]] += Sfssf;
		igGrad[neighbour[facei]] -= Sfssf;
	}
	igGrad /= mesh.V();

	gGrad.correctBoundaryConditions();
	...
}
\end{lstlisting}

%%%%%%%%%%%%%%
\section{Linear solvers}
Iterative solvers in OpenFOAM can be divided into for symmetric matrices and asymmetric matrices. The former is results from discretisation of time dependent term and diffusion term, while the latter is from discretisation of convection term. Selection of solver uses \bera{solver} keyword with options are:
\begin{itemize}
	\item \bera{diagonal}: diagonal solver for both symmetric and asymmetric matrices using for explicit systems.
	\item \bera{smoothSolver}: solver that uses a smoother for symmetric and asymmetric matrices with a run-time selected smoother.
	\item \bera{PCG}: preconditioned conjugate gradient for symmetric matrices.
	\item \bera{PBiCG/PBiCGStab}: (Stabilised) preconditioned bi-conjugate gradient with run-time selectable preconditioner for asymmetric matrices.
	\item \bera{GAMG}: generalised geometric-algebraic multi-grid solver.
\end{itemize}

\subsection{Solution tolerances}
Iterative solver reduces equation residuals after each iteration. Residuals are normalized to represent error of the solution. In OpenFOAM, three parameters are used to terminate iterations in each time step. \bera{tolerance} represents a solver tolerance value to stop the solver when the residual reaches the value. \bera{relTol} is a relative tolerance defined as the ratio of current over initial residuals. \bera{maxIter} defines the maximum number of iterations. Equations can be solved several times within a time step. In this case, different setting of solutions tolerances can be used, i.e. set solver tolerance just for the last iteration, and relative tolerances for others.

\subsection{Preconditioner}
For conjugate gradient solvers, preconditioning options can be divided into those used for symmetric matrix: such as diagonal incomplete-Cholesky (\bera{DIC}), Faster Diagonal-based Incomplete Cholesky (\bera{FDIC}); for asymmetric matrix: such as diagonal incomplete-LU (\bera{DILU}) or for both diagonal preconditioning (\bera{diagonal}); Geometric agglomerated Algebraic MultiGrid (\bera{GAMG}).

\begin{algorithm}[H]
	\caption{Conjugate gradient method} \label{alg:CG-method}
	\begin{algorithmic}[1]
		\STATE {Initialize iteration index: $k=0$, initial solution: $\phi^0 = \phi_{in}$, initial residual: $\vb*{\rho}^0 = \vb*{Q} - A\phi_{in}$, initial direction: $\vb*{p}^0=\vb*{0}$, $s_0 = 10^{30}$}
		\STATE {$k=k+1$}
		\STATE {Solving: $M \vb*{z}^k = \vb*{\rho}^{k-1}$}
		\FOR {Construct new solution, residual, search direction: }
		\STATE {$s^k = \vb*{\rho^{k-1}} \cdot \vb*{z^k}$}
		\STATE {$\beta^k = s^k/s^{k-1}$}
		\STATE {$\vb*{p}^k = \beta^k \vb*{p}^{k-1}$}
		\STATE {$\alpha^k = s^k/(\vb*{p}^{k} \cdot A \vb*{p}^{k})$}
		\STATE {$\phi^k = \phi^{k-1} + \alpha^k \vb*{p}^{k}$}
		\STATE {$\rho^k = \rho^{k-1} - \alpha^k A \vb*{p}^{k}$}
		\ENDFOR
		\STATE {Repeat until desired residual is reached}
	\end{algorithmic}
\end{algorithm}

\subsection{Generalised Geometric-Algebraic Multi-Grid solver (GAMG)}
Multi-grid solvers use a fast solution from coarse grid to eliminate high frequency errors and use this for finer grid. \bera{faceAreaPair} is a agglomeration algorithm used in OpenFOAM for coarsening the mesh. A simple two-grid iteration method algorithm can be presented in Algorithm~\ref{alg:Two-grid-method}.
\begin{algorithm}[H]
	\caption{Two-grid iteration method} \label{alg:Two-grid-method}
	\begin{algorithmic}[1]
		\STATE {Perform iterations on the fine grid}
		\STATE {Compute residual on fine grid}
		\STATE {Restrict residual to the coarse grid}
		\STATE {Perform iterations of correction equation on the coarse grid}
		\STATE {Interpolate correction to the fine grid}
		\STATE {Update correction on the fine grid}
		\STATE {Repeat until desired residual is reached}
	\end{algorithmic}
\end{algorithm}

\section{Boundary conditions}
\textcite{Moukalled2016} highlighted the differences between physical conditions, e.g. "wall", "inlet", "outlet"; geometric constraints, e.g. "symmetry", "periodic"; and boundary conditions, i.e. set of equations used to define the variables at a domain boundary. For each physical condition, many derived types of boundary conditions can be imposed. Geometric constrains are applied to reduce the domain size and usually relate to a specific boundary conditions, such as "symmetry" condition implies zero normal flux along the specified boundary. Boundary conditions can be classified into three types: "Dirichlet condition", where value of variables is defined; "von Neumann condition", where flux of variables is defined and "Robin condition", where variables and flux are derived from constitutive equations.

Dirichlet and von Neumann are the standard boundary condition types for all fields in OpenFOAM. Besides, derived boundary conditions can be used to set boundary conditions for fields from a derived field, e.g. \bera{fixedShearStress} is used to set a constant shear stress $\tau_0$ for velocity field $U$ as:
\begin{equation}
\tau_0 = -\nu_{eff} \frac{dU}{dn}
\end{equation}
where $\nu_{eff}$ is kinematic viscosity and $n$ denotes the surface normal.

\section{Thermophysical models}
Energy, heat and transport properties are determined by a set of thermophysical models \cite{Greenshields2017} in OpenFOAM. This set is defined in a dictionary, call \bera{thermoType}. An example of \bera{thermoType} dictionary which defines mixture type, transport and thermodynamic properties models, choice of energy equation variable and equation of states is presented in Listing~\ref{lst:ThermoTypeDict}.
\begin{lstlisting}[caption={ThermoType dictionary}, label={lst:ThermoTypeDict}]
ThermoType
{
	type            heRhoThermo;
	mixture         reactingMixture;
	transport       sutherland;
	thermo          janaf;
	energy          sensibleEnthalpy;
	equationOfState perfectGas;
	specie          specie;
}
\end{lstlisting}
The fluid in a simulation can be classified as single composition and mixture of fixed or variable compositions. Two choices of energy equation variables are either internal energy or enthalpy. Transport and thermodynamic properties are determined using models based on the compressibility $\psi = \left ( RT \right )^{-1}$ ($R$ is universal gas constant) or the density $\rho$, which are calculated from pressure and temperature fields.

Transport models are used to calculate transport variables such as dynamic viscosity $\mu$, thermal conductivity $\kappa$ and thermal diffusivity $\alpha$. \bera{const} model assumes constant transport properties. \bera{sutherland} model uses Sutherland's formula to define transport properties as a function of temperature:
\begin{equation} \label{eq:sutherlandTransport}
\mu = A_s \frac{\sqrt{T}}{1 + T_s / T}
\end{equation}
where $A_s$ is Sutherland coefficient and $T_s$ is Sutherland temperature.
A polynomial function of order $N$ can be used to relate transport properties with temperature field (\bera{polynomial} model):
\begin{equation} \label{eq:foam_tranPoly}
\mu = \sum_{i}^{N-1} a_i  T^{i}
\end{equation}
where $a_i$ is a coefficient of the polynomial.

The thermodynamic models are used to calculate the specific heat $c_p$ (at constant pressure) of the fluid. $c_p$ can be assumed to take a constant value using \bera{hConst} model. \bera{hPolynomial} model uses a $N$th polynomial function of temperature to define values of $c_p$ as:
\begin{equation} \label{eq:foam_hPoly}
c_p = \sum_{i}^{N-1} a_i  T^{i}
\end{equation}
where $a_i$ is a coefficient of the polynomial.

Equation of state is used to derive density field in OpenFOAM. \bera{perfectGas} model uses ideal gas law to relate fluid density $\rho$ with its pressure and temperature: $\rho = p/(RT)$. Other option is \bera{icoPolynomial}, which define density as a $N$th polynomial function of temperature:
\begin{equation} \label{eq:foam_icoPoly}
\rho = \sum_{i}^{N-1} a_i  T^{i}
\end{equation}
where $a_i$ is a coefficient of the polynomial.

\section{Turbulence models}
\subsection{Reynolds-Averaged Navier-Stokes (RANS)}
Reynolds-Averaged Navier-Stokes (RANS) equations derived by averaging Navier-Stokes equations are listed below \cite{Ferziger2012}:
\begin{equation} \label{eq:rans-continuity}
\frac{\partial}{\partial x_i} \left( \rho \bar{u_i} \right) = 0
\end{equation}
\begin{equation} \label{eq:rans-momentum}
\frac{\partial}{\partial t} \left( \rho \bar{u_i} \right)
+ \frac{\partial}{\partial x_j} (\rho \bar{u_i} \bar{u_j})
=
-\frac{\partial \bar{p}}{\partial x_i}
+ \frac{\partial \tau_{ij}}{\partial x_j}
+ \frac{\partial}{\partial x_j} (-\rho \overline{u'_i u'_j})
+ \rho g_i
\end{equation}

For a scalar $\phi$, Reynolds-Averaged governing equation can be written as:
\begin{equation} \label{eq:rans-scalar}
\frac{\partial}{\partial t} \left(\rho \bar{\phi} \right)
+ \frac{\partial}{\partial x_j}(\rho \bar{u_j} \bar{\phi})
=
\frac{\partial}{\partial x_j} (\Gamma \frac{\partial \bar{\phi}}{\partial x_j})
+\frac{\partial}{\partial x_j} (-\rho \overline{u'_j \phi'})
\end{equation}

Turbulent fluxes are modelled using eddy-viscosity hypothesis which treat turbulent fluxes similar to laminar ones:
\begin{equation} \label{eq:momentum turbulent flux}
-\rho \overline{u'_i u'_j}
=
\mu_t \left( \frac{\partial u_i}{\partial x_j} + \frac{\partial u_j}{\partial x_i} \right)
- \frac{2}{3} \rho k \delta_{ij}
\end{equation}

\begin{equation} \label{eq:energy turbulent flux}
-\rho \overline{u'_i {\phi}'}
=
\Gamma_t \frac{\partial \phi}{\partial x_j}
\end{equation}

\begin{equation} \label{eq:species turbulent flux}
-\rho \overline{u'_i \omega'_\alpha}
=
\Gamma_{t\alpha} \frac{\partial \omega_\alpha}{\partial x_j}
\end{equation}

Turbulent diffusivity $\Gamma_t$ can be defined from turbulence Prandtl number $Pr_{t}$:
\begin{equation} \label{eq:turb-diffusivity}
\Gamma_t =  \frac{\mu_t}{Pr_{t}}
\end{equation}

\subsubsection{The $k-\epsilon$ model}
The $k-\epsilon$ model relies on Prandtl-Kolmogorov expression \cite{Busini2016} as:
\begin{equation} \label{eq:turb-viscosity}
\mu_t=\rho C_\mu \frac{k^2}{\epsilon}
\end{equation}

Two additional transport equations for turbulence kinetic energy and turbulence dissipation rate are required. In OpenFOAM, standard version from \cite{Launder1974} is implemented. The transport equations for $k$ and $\epsilon$ are:
\begin{equation} \label{eq:foam_k}
\frac{D}{D t} (\rho k) =
\div \left(\rho D_k \grad k\right) + G_k - \frac{2}{3}\rho \left(\div \vb*{u}\right) k - \rho \epsilon + S_k
\end{equation}

\begin{equation} \label{eq:foam_epsilon}
\frac{D}{D t} (\rho \epsilon) =
\div \left(\rho D_{\epsilon} \grad \epsilon \right) + \frac{C_{1\epsilon} G_k \epsilon}{k} - \left(\frac{2}{3}C_{1\epsilon} - C_{3,RDT}\right)\rho \left(\div \vb*{u} \right) \epsilon - C_{2\epsilon} \rho \frac{\epsilon^2}{k} + S_\epsilon
\end{equation}

The model constants are tabulated in Table~\ref{tab:kEpsCons}. $C_{3,RDT}$ is the rapid distortion theory (RDT) based compression term with can be excluded by setting the constant to zero.
\def\rowWidth{0.06}
\begin{table}[htbp]
	\caption{The standard $k-\epsilon$ model constants} \label{tab:kEpsCons}
	\centering
	\begin{tabular}{p{\rowWidth\textwidth}p{\rowWidth\textwidth}p{\rowWidth\textwidth}p{\rowWidth\textwidth}p{\rowWidth\textwidth}}
		\toprule
		$C_{1\epsilon}$	& $C_{2\epsilon}$ & $C_{\mu}$ & $\sigma_k$ & $\sigma_\epsilon$ \\
		\midrule
		1.44 & 1.92	& 0.09 & 1 & 1.3\\
		\bottomrule
	\end{tabular}
\end{table}

To including the effect of buoyancy, the transport equations for $k$ and $\epsilon$ are:
\begin{equation} \label{eq:turb-kinetic-energy-eq}
\frac{D}{D t} (\rho k)
=
\frac{\partial}{\partial x_i} \left[ \left( \mu + \frac{\mu_t}{\sigma_k} \right) \frac{\partial k}{\partial x_j} \right] + G_k + G_b
- \rho \epsilon
\end{equation}

\begin{equation} \label{eq:dissipation-rate-eq}
\frac{D}{D t} (\rho \epsilon)
=
\frac{\partial}{\partial x_i} \left[ \left( \mu + \frac{\mu_t}{\sigma_\epsilon} \right) \frac{\partial \epsilon}{\partial x_j} \right]
+ C_{1\epsilon} \frac{\epsilon}{k} G_k
+ C_{1\epsilon} C_{3\epsilon} \frac{\epsilon}{k} G_b
- C_{2\epsilon} \rho \frac{\epsilon^2}{k}
\end{equation}
where $G_k$ is production of turbulence kinetic energy due to the mean velocity gradients:
\begin{equation}
G_k = \mu_t S^2
\end{equation}

$G_b$ is the buoyancy source term:
\begin{equation}
G_b = -\frac{\mu_t}{\rho Pr_t} (\vb*{g} \cdot \grad{\rho}) = - C_g \nu_t (\vb*{g} \cdot \grad{\rho})
\end{equation}

\textcite{Chan1997} proposed model constants as: $C_{1\epsilon} = 1.44$, $C_{2\epsilon} = 1.92$, $C_{3\epsilon}=-0.8$ and $C_{3\epsilon} = 2.15$ for the unstable and stable regimes respectively.

In case of undefined value of $C_{3\epsilon}$, OpenFOAM use the default value of $C_{3}$ calculated using:
\begin{equation}
C_{3\epsilon} = tanh \left| \frac{v}{u} \right|
\end{equation}

An example of model coefficients setting, and code for solving turbulence using standard $k-\epsilon$ model are listed in Listing~\ref{lst:kEpsilon}.
\lstinputlisting[caption={k, $\epsilon$ transport equation implementation in OpenFOAM}, label=lst:kEpsilon]{codes/kEpsilon.C}

The inclusion of the buoyancy term in standard $k-\epsilon$ is implemented in OpenFOAM as in Listing~\ref{lst:buoyantKEpsilon}.
\lstinputlisting[caption={Implementation of including buoyancy source term in OpenFOAM}, label=lst:buoyantKEpsilon]{codes/buoyantKEpsilon.C}

\subsubsection{The SST $k-\omega$ model}
The SST $k-\omega$ model \cite{Menter2003TenYO} is shown to effectively solve turbulence in strong adverse pressure gradients and separation, which is failed when using standard $k-\epsilon$ model.
\begin{equation}
\frac{D}{Dt}{\rho k} = \div \left( \rho (\mu + \alpha_{k} \mu_t) \grad k \right) + \rho G - \frac{2}{3} \rho k \left( \div \vb*{u} \right) - \rho \beta^{*} \omega k + S_k.
\end{equation}

\begin{equation}
\frac{D}{Dt}(\rho \omega) = \div \left( \rho (\mu + \alpha_{\omega} \mu_t) \grad \omega \right) + \rho \gamma \frac{G}{\nu} - \frac{2}{3} \rho \gamma \omega \left( \div \vb*{u} \right) - \rho \beta \omega^2 - \rho \left(F_1 - 1 \right) CD_{k\omega} + S_\omega
\end{equation}
$G$ is turbulent production term calculated similar to standard $k-\epsilon$ model. The blending function $F_1$ is the trick to proper selection of $k-\omega$ and $k-\epsilon$ without the user interaction, which is 0 away from the surface and switch to 1 inside the boundary layer:
\begin{equation}
F_1 = \tanh \left( \left( \min \left[ \max \left( \frac{k^{1/2}}{\beta^* \omega y}, \frac{500 \nu}{y^2 \omega}\right) \frac{4 \rho \alpha_{\omega 2} k}{CD_{k\omega} y^2} \right]\right) ^4 \right)
\end{equation}

\begin{equation}
CD_{k\omega} = \max \left( 2 \rho \alpha_{\omega 2} \frac{1}{\omega} \pdv{k}{x_i} \pdv{\omega}{x_i} , 10 ^{-10} \right)
\end{equation}

The turbulent eddy viscosity:
\begin{equation}
\nu_t = a_1 \frac{k}{\max (a_1 \omega_, b_1 F_{2} \sqrt{2 S_{ij} S_{ij}})}
\end{equation}
$F_{2}$ is the second blending function:
\begin{equation}
F_2 = \tanh \left(\left[ \max \left( \frac{ 2 k^{1/2}}{\beta^* \omega y}, \frac{500 \nu}{y^2 \omega}\right) \right]^2 \right)
\end{equation}

All constants are calculated from blended coefficient $F_1$, e.g. $\alpha_{k} = \alpha_{k1} F_1 + \alpha_{k2} (1 - F_1)$. The constants table is given in Table~\ref{tab:kOmeCons}.
\def\rowWidth{0.06}
\begin{table}[htbp]
	\caption{The $k-\omega$ model constants} \label{tab:kOmeCons}
	\centering
	\begin{tabular}{p{\rowWidth\textwidth}p{\rowWidth\textwidth}p{\rowWidth\textwidth}p{\rowWidth\textwidth}p{\rowWidth\textwidth}p{\rowWidth\textwidth}p{\rowWidth\textwidth}p{\rowWidth\textwidth}p{\rowWidth\textwidth}p{\rowWidth\textwidth}p{\rowWidth\textwidth}p{\rowWidth\textwidth}}
		\toprule
		$\alpha_{k1}$ & $\alpha_{k2}$ & $\alpha_{\omega 1}$	& $\alpha_{\omega 2}$ & $\beta_1$ & $\beta_2$ & $\gamma_1$ & $\gamma_2$ & $\beta^*$ & $a_1$ & $b_1$ & $c_1$ \\
		\midrule
		0.85 & 1.0 	& 0.5 	& 0.856 & 0.075 & 0.0828 & 5/9 & 0.44 & 0.09 & 0.31 & 1.0 & 10.0 	\\
		\bottomrule
	\end{tabular}
\end{table}

\subsubsection{Wall function}
The near-wall region can be divided into three parts: the viscous sub-layer, the buffer layer and the logarithmic region. The extend of the logarithmic region is increased with increasing Reynold number. Resolving the flow near boundary layer requires generally an excessive number of computational cells, while the adjacent cells of the boundary are required to be in viscous layer, i.e. the wall length scale $y^+=1$. The wall function is used to overcome this restriction, by proposing a boundary condition at logarithmic region ($30 < y^{+} < 200$). This is not only help to increase cell size of wall adjacent cell but also improve grid maximum aspect ratio, which in turn improving computational stiffness \cite{Kalitzin2005}.

The derivations of wall functions are rooted from the generality of the region between the wall and the outer edge of the logarithmic layer in the quasi-equilibrium boundary layer, e.g. flow over a flat plate at zero-pressure gradient \cite{Kalitzin2005}. The RANS equations are simplified in this case as:
\begin{equation}
\frac{d}{dy}\left( (\mu +\mu_t) \frac{dU}{dy} \right) = 0
\end{equation}

Appropriate scales using for near wall region are:
\begin{equation}\label{eq:wallScales}
u^{+}=\frac{U}{u_*}, y^{+}=\frac{y u_*}{\nu}, \nu_t^{+}=\frac{\nu_t}{\nu}, k^{+} = \frac{k}{u_*^2}, \epsilon^{+} = \frac{\epsilon \nu}{u_*^4}, \omega^{+} = \frac{\omega \nu}{u_*^2}
\end{equation}

For RANS models, in viscous sub-layer ($y^{+} < 5$), the flow is dominated by viscous effect. Therefore, the fluid shear stress is balanced with the wall shear stress. The velocity is linearly evolved from the wall.
\begin{equation}
u^{+} = y^{+}
\end{equation}

In logarithmic region, the flow is dominated by turbulence stress. Assuming the sum of the viscous and turbulent shear stress is constant and equal to the wall shear stress, Prandtl's assumption for the turbulent viscosity $\nu_t^{+} = \kappa y^{+}$, velocity can be derived as:
\begin{equation}\label{eq:wallVelLog}
u^{+} = \frac{1}{\kappa} \ln(Ey^{+})
\end{equation}
$\kappa = 0.41$ is the von-Karman constant, $E$ is a constant adjusted by empirical data. Calculation of $u_*$ can be done from solving wall function (given velocity of first cell) or from solving momentum equation \cite{Kalitzin2005}. Eddy-viscosity $\nu_t$ is implicitly related to the velocity profile (through $\kappa$), therefore it may be used to derive a boundary condition for turbulence variables to ensure consistency.

For $v^2-f$ turbulence model, by applying turbulence kinetic energy equation for near wall region using non-dimensional variables defined in Equation~(\ref{eq:wallScales}, we have relation for logarithmic layer:
\begin{equation}\label{eq:wallKLog}
k^{+} = \frac{C_k}{\kappa} \ln(y^{+}) + B_k
\end{equation}
The constant $C_k=-0.416$, $B_k=8.366$ are used in OpenFOAM.

For viscous sub-layer:
\begin{equation}\label{eq:wallKVis}
k^{+} = \frac{2400}{C_{2\epsilon}} \left[ \frac{1}{(y^{+}+C)^4} + \frac{2y^{+}}{C^3} - \frac{1}{C^2} \right]
\end{equation}
These two regimes are separated using the value of $y^+_{laminar}$
\begin{equation}\label{eq:wallYPlusLam}
y^+_{laminar} = \frac{\ln(\max (E y^+_{laminar},1))}{\kappa}
\end{equation}

Similarly, for $\epsilon$ variable, we have relation for logarithmic layer:
\begin{equation}\label{eq:wallEpsLog}
\epsilon^{+} = \frac{1}{\kappa y^{+}}
\end{equation}
For viscous sub-layer:
\begin{equation}\label{eq:wallEpsVis}
\epsilon^{+} = 2\frac{k^{+}}{(y^{+})^2}
\end{equation}

For $\omega$ variable, we have relation for logarithmic layer:
\begin{equation}\label{eq:wallOmeLog}
\omega^{+} = \frac{1}{\kappa C_\mu^{1/2} (y^{+})^2}
\end{equation}
For viscous sub-layer ($\beta_1$ is the $k-\omega$ model constant):
\begin{equation}\label{eq:wallOmeVis}
\omega^{+} = \frac{6}{\beta_1 (y^{+})^2}
\end{equation}

% k wall function
Two standard implementations of wall function in OpenFOAM for turbulence kinetic energy $k$ are \bera{kqRWallFunction}, which is Neumann boundary (zero gradient), and \bera{kLowReWallFunction}, which is fixed value condition. The value of boundary patches is calculated from Equation~(\ref{eq:wallKVis}) or Equation~(\ref{eq:wallKLog}), where important parameter $y^+$ is defined using known value of  the friction velocity $u_*$. $u_*$ can be calculated from simple relation derived by \textcite{Launder1974}, assuming that generation and dissipation of energy are in balance:
\begin{equation}\label{eq:wallFrictionVel}
u_* = C_\mu^{1/4} k^{1/2}
\end{equation}
$k$ is turbulence kinetic energy in the first cell adjacent to the wall. Value of $k$ in wall patches is calculated from $k^{+}$ as:
\begin{equation}
k = k^{+} u_*^2
\end{equation}

% \epsilon and \omega wall function
Unlike $k$ wall function, which value is defined at boundary surface, $\epsilon$ and $\omega$ wall function define its values in cell centre. Using \bera{epsilonWallFunction}, the value is averaged from all surfaces defined as wall in the cell:
\begin{equation}
\epsilon_p = \frac{C_{\mu}^{0.75} k^{1.5}}{\kappa y_P}
\end{equation}

% nu_t wall functions
$\nu_t$ wall functions are also necessarily defined in OpenFOAM. It is used to define wall shear stress as a remedy to below approximation where wall velocity gradient is significantly larger than velocity difference between the adjacent cell and the wall:
\begin{equation}
\tau_w=\nu \pdv{u}{n} |_w \approx \nu \frac{(u_P - u_w)}{y_P}
\end{equation}

Equation for the wall shear stress can be derived using Equation~(\ref{eq:wallFrictionVel}) and (\ref{eq:wallVelLog}):
\begin{equation}
\tau_w=\rho u_*^2 = \rho u_* \frac{(u_P - u_w)}{\frac{1}{\kappa}\ln (E y^+)}
\end{equation}
$\nu_t$ wall functions is derived from the above two equations as:
\begin{equation} \label{eq:wallNut}
\nu_t=\nu \left( \frac{\kappa y^+}{\ln (E y^+)} -1 \right)
\end{equation}

$\nu_t$ wall functions is then used to implicitly define velocity at wall adjacent cell. There are several options to use $\nu_t$ wall functions in OpenFOAM. \bera{nutLowReWallFunction} set $\nu_t = 0$ which means that the flow near wall is sufficiently solved. \bera{nutkWallFunction} and \bera{nutUWallFunction} both use Equation~(\ref{eq:wallNut}) but with different calculation of $y^+$. The former uses assumption of Eqn~\ref{eq:wallFrictionVel} for the calculation of $y^+$ as:
\begin{equation}
y^+ = \frac {y C_\mu^{1/4} k^{1/2}}{\nu}
\end{equation}

\bera{nutUWallFunction} using Equation~(\ref{eq:wallVelLog}) to derive relationship between $y^+$ and $u_P$:
\begin{equation}
y^+ \ln (E y^+) - \frac{\kappa y u_P}{\nu} = 0
\end{equation}
The above equation is solver using Newton-Raphson iterative method to find $y^+$.

\bera{nutkAtmRoughWallFunction} calculate turbulent viscosity at wall adjacent cell as:
\begin{equation}
\nu_t = \frac{\kappa u_* y_P}{\left(\ln \frac{y_P + z_0}{z_0}\right)}
\end{equation}

In case of sand-grain type rough wall model, implemented in \bera{nutkRoughWallFunction} class, wall roughness value affects drag (resistance), heat and mass transfer on the wall. The wall velocity scale can be written as \cite{Cavar2016}:
\begin{equation}
u^+=\frac{1}{\kappa}\ln(Ey^+)-\Delta B
\end{equation}
\begin{equation}
\Delta B = \frac{1}{\kappa}\ln(f_r)
\nonumber
\end{equation}
$f_r$ quantifies the shift of the intercept due to roughness effects. For fully rough region:
\begin{equation}
f_r = 1 + C_s K_s^{+}
\end{equation}

The roughness constant $C_s$ is used to model the roughness effect. A proper roughness constant is dictated mainly by the type of given roughness. The default roughness constant was retained as $C_s=0.5$, which indicating tightly-packed, uniform sand-grain. The non-dimensional sand-grain roughness height $K_s^{+}$ is:
\begin{equation}
K_s^{+} = \frac{\rho K_s u_*}{\mu}
\end{equation}
\begin{equation}
u_* = C_{\mu}^{1/4}k_p^{1/2}
\nonumber
\end{equation}
$\nu_t$ is calculated using:
\begin{equation}
\nu_t=\frac{u_* \kappa y_P}{\ln \left( \frac{E y_P}{C_s k_s}\right)}
\end{equation}

\subsection{Reynold stress transport models}
Launder, Reece and Rodi (LRR) turbulence model \cite{Gibson1978} is the most commonly used Reynolds stress transport model. Individual Reynolds stresses are solved instead of using turbulent viscosity models, e.g. Equation \ref{eq:turb-viscosity}.  Solving conservation equations for each individual Reynolds stresses is costly, but overcome deficiency due to modelling of turbulent viscosity. The conservation equations are given as:
\begin{equation} \label{eq:lrr_eqns}
\frac{D}{D t} (\tau_{ij}) = -\tau_{im} \pdv{u_j}{x_m} +\tau_{jm} \pdv{u_i}{x_m}+\rho \epsilon_{ij} - \Pi_{ij} + \pdv{}{x_m} \left( \nu_l \pdv{\tau_{ij}}{x_m} +C_s\frac{2k^2}{3 \epsilon}\pdv{\tau_{ij}}{x_k} \right)
\end{equation}

Dissipation term: $\epsilon_{ij}=2/3 \epsilon \delta_{ij}$. Pressure strain term comprises two components, $\Pi_{ij}=A_{ij} + M_{ijmn}\pdv{u_m}{x_n}$. The term involving fluctuation quantities:
\begin{equation}
A_{ij} = C_{S1}\frac{\epsilon}{k} \left( \tau_{ij} + \frac{2}{3} \rho k \delta_{ij} \right)
\end{equation}
And the term involving mean rate of strain:
\begin{equation}
M_{ijmn} = -\hat{\alpha} \left( P_{ij} - \frac{1}{3} P_{kk} \delta_{ij} \right) -\hat{\beta} \left( D_{ij} - \frac{1}{3} D_{kk} \delta_{ij} \right) - \hat{\gamma}\rho k \S_{ij}
\end{equation}
\begin{equation}
\begin{aligned}
P_{ij} = \tau_{ik} \pdv{u_j}{x_k} +\tau_{jk} \pdv{u_i}{x_k}\\
D_{ij} = \tau_{ik} \pdv{u_k}{x_j} +\tau_{jk} \pdv{u_k}{x_i}\\
S_{ij} = \frac{1}{2} \left(\pdv{u_i}{x_j} + \pdv{u_j}{x_i}\right)\\
\end{aligned}
\end{equation}
$\hat{\alpha}$, $\hat{\beta}$, $\hat{\gamma}$ are calculated in Table~\ref{tab:LRRconstant}:
\begin{table}[h!]
	\caption{The LRR mean rate of strain term constants}
	\label{tab:LRRconstant}
	\centering
	\begin{tabular}{ccc}
		\toprule
		$\hat{\alpha}$ & $\hat{\beta}$ & $\hat{\gamma}$ \\
		\midrule
		$({8+C_2})/{11}$ & $({8C_2-2})/{11}$& $({60C_2-4})/{55}$\\
		\bottomrule
	\end{tabular}
\end{table}

Equation of turbulence energy dissipation rate is solved for the dissipation term \cite{Launder1975}:
\begin{equation} \label{eq:llr_dissipation_eq}
\frac{D}{D t} (\epsilon)
=
C_\epsilon \frac{\partial}{\partial x_i} \left[ \frac{k}{\epsilon} \overline{u_i u_j}\frac{\partial \epsilon}{\partial x_j} \right]
- C_{\epsilon 1} \frac{\epsilon}{k} G_k
- C_{\epsilon 2} \rho \frac{\epsilon^2}{k}
\end{equation}

Model constants of the quasi-isotropic variant of this model are:
\begin{table}[h!]
	\caption{The LRR model constants} \label{tab:lrr}
	\centering
	\begin{tabular}{cccccc}
		\toprule
		$C_{s}$	& $C_{S1}$ & $C_{2\epsilon}$ & $C_{\epsilon 1}$ & $C_{\epsilon 2}$ & $C_\epsilon$ \\
		\midrule
		0.22 & 1.8	& 0.4 & $1.44$ & $1.92$& $0.15$\\
		\bottomrule
	\end{tabular}
\end{table}

The default model constants in OpenFOAM are revealed in Listing~\ref{lst:lrrConst}.
\begin{lstlisting}[caption={LRR model constants selection},label=lst:lrrConst]
LRRCoeffs
{
	Cmu             0.09;
	C1              1.8;
	C2              0.6;
	Ceps1           1.44;
	Ceps2           1.92;
	Cs              0.25;
	Ceps            0.15;

	wallReflection  yes;
	kappa           0.41
	Cref1           0.5;
	Cref2           0.3;

	couplingFactor  0.0;
}
\end{lstlisting}

\section{The solvers}
%%%%%%%%%%%%
\paragraph{\bera{buoyantNonReactingSimpleFoam}}
A solver, \bera{buoyantNonReactingSimpleFoam} is developed from \bera{buoyantSimpleFoam} and \bera{rhoReactingBuoyantFoam} for steady state simulation of ABL. While the solution of steady state simulation of ABL is used as initial condition for simulation gas dispersion. \bera{buoyantNonReactingSimpleFoam} is compatible with \bera{buoyantNonReactingPimpleFoam} which is used to simulate transient gas dispersion in ABL. Thermalphysical model is a density based thermodynamics package with non-reacting mixture of air and gas. Enthalpy $h$ is chosen as energy conservation variable. Buoyancy is taken into account by solving \bera{p\_rgh} field defined in Equation~(\ref{eq:foam_p_rgh}) instead of \bera{p}. Velocity-pressure coupling is solved using $SIMPLE$ algorithm. The code is listed below:
\lstinputlisting[caption={\bera{buoyantNonReactingSimpleFoam} main code}]{codes/buoyantSimpleFoam.C}

\paragraph{\bera{buoyantNonReactingPimpleFoam}}
\bera{buoyantNonReactingPimpleFoam} is developed from \bera{rhoReactingBuoyantFoam}. The specie transport equation is modified to apply user-defined turbulent Schmidt number. Taking into account buoyancy effect, \bera{p\_rgh} field defined in Equation~(\ref{eq:foam_p_rgh}) is solved instead of \bera{p}. $PIMPLE$ algorithm is used to solve velocity-pressure coupling.

The code is listed below:
\lstinputlisting[caption={\bera{buoyantNonReactingPimpleFoam} main code}]{codes/rhoReactingBuoyantFoam.C}

\paragraph{\bera{rhoEqn}} is used to solve the continuity for density:
\lstinputlisting[caption={\bera{rhoEqn.H}}]{codes/rhoEqn.H}

\paragraph{\bera{YEqn}} is modified with user-defined $Sc_t$:
\lstinputlisting[caption={\bera{YEqn.H}}]{codes/YEqnModified.H}
