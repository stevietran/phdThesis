% ******************************************************************************
% ****************************** Custom Margin *********************************

% Add `custommargin' in the document class options to use this section
% Set {innerside margin / outerside margin / topmargin / bottom margin}  and
% other page dimensions
\ifsetCustomMargin
  \RequirePackage[left=37mm,right=30mm,top=35mm,bottom=30mm]{geometry}
  \setFancyHdr % To apply fancy header after geometry package is loaded
\fi

% Add spaces between paragraphs
%\setlength{\parskip}{0.5em}

% Ragged bottom avoids extra whitespaces between paragraphs
\raggedbottom

% To remove the excess top spacing for enumeration, list and description
%\usepackage{enumitem}
%\setlist[enumerate,itemize,description]{topsep=0em}


% *****************************************************************************
% ******************* Fonts (like different typewriter fonts etc.)*************
% Add `customfont' in the document class option to use this section

\ifsetCustomFont
  % Set your custom font here and use `customfont' in options. Leave empty to
  % load computer modern font (default LaTeX font).
  %\RequirePackage{helvet}

  % For use with XeLaTeX
  %  \setmainfont[
  %    Path              = ./libertine/opentype/,
  %    Extension         = .otf,
  %    UprightFont = LinLibertine_R,
  %    BoldFont = LinLibertine_RZ, % Linux Libertine O Regular Semibold
  %    ItalicFont = LinLibertine_RI,
  %    BoldItalicFont = LinLibertine_RZI, % Linux Libertine O Regular Semibold Italic
  %  ]
  %  {libertine}
  %  % load font from system font
  %  \newfontfamily\libertinesystemfont{Linux Libertine O}
\fi

%%%%%%%%%%%%%%%%%%%%%%%%%%%%%%%%%%%%
% Custom Packages
%%%%%%%%%%%%%%%%%%%%%%%%%%%%%%%%%%%%

% For Figure
\usepackage{graphicx}
\usepackage{epstopdf}
% Chemical formular
\usepackage{mhchem}

% Math fonts
\usepackage{amssymb,bm}
\usepackage[italicdiff]{physics}	% differential symbol d into italic form
% \div, \grad, \dd, \vb, \vdot
% use this package module for SI units
\usepackage{siunitx} 

% Algorithms and Pseudocode 
% beramono
\edef\oldtt{\ttdefault}
\usepackage[scaled]{beramono}
\usepackage[T1]{fontenc}
\newcommand{\bera}[1]{{\fontfamily{fvm}\selectfont #1\hyphenchar\font=45}}
\renewcommand*\ttdefault{\oldtt}

% Algorithms
\usepackage{algorithm,algorithmic}

% Code listing
\usepackage{listings}
\usepackage{color}
\definecolor{codegreen}{rgb}{0,0.6,0}
\definecolor{codegray}{rgb}{0.5,0.5,0.5}
\definecolor{codepurple}{rgb}{0.58,0,0.82}
\definecolor{backcolour}{rgb}{0.95,0.95,0.92}
\lstdefinestyle{mystyle}{
	language=C++,
	backgroundcolor=\color{backcolour},   
	commentstyle=\color{codegreen},
	keywordstyle=\color{magenta},
	numberstyle=\tiny\color{codegray},
	stringstyle=\color{codepurple},
	%basicstyle=\footnotesize\ttfamily,
	basicstyle=\scriptsize \fontfamily{fvm}\selectfont,
	breakatwhitespace=false,         
	breaklines=true,                 
	captionpos=tl,                    	                 
	numbers=left,                    
	numbersep=5pt,                  
	showspaces=false,                
	showstringspaces=false,
	showtabs=false,                  
	tabsize=4
}
\lstset{style=mystyle}
\lstset{columns=flexible}
\lstset{keepspaces=true}
%%%%%%%%%%%%%%%%%%%%%%%%%%%%%%%%%%%%%%%%%%%%
% Graphics and figures
%%%%%%%%%%%%%%%%%%%%%%%%%%%%%%%%%%%%%%%%%%%%
% Captions: This makes captions of figures use a boldfaced small font.
%\RequirePackage[small,bf]{caption}
\RequirePackage[labelsep=space,tableposition=top]{caption}
\renewcommand{\figurename}{Fig.} %to support older versions of captions.sty
%\usepackage{rotating}
%\usepackage{wrapfig}

% Uncomment the following two lines to force Latex to place the figure.
% Use [H] when including graphics. Note 'H' instead of 'h'
%\usepackage{float}
%\restylefloat{figure}

% Subcaption package is also available in the sty folder you can use that by
% uncommenting the following line
% This is for people stuck with older versions of texlive
%\usepackage{sty/caption/subcaption}
\usepackage{subcaption}

%%%%%%
% Table
%%%%%%
\usepackage{booktabs} % For professional looking tables
\usepackage{multirow}
%\usepackage{multicol}
%\usepackage{longtable}
%\usepackage{tabularx}


% ******************************* Line Spacing *********************************
% Choose linespacing as appropriate. Default is one-half line spacing as per the
% University guidelines

\doublespacing
%\onehalfspacing
% \singlespacing


% ************************ Formatting / Footnote *******************************
% Don't break enumeration (etc.) across pages in an ugly manner (default 10000)
%\clubpenalty=500
%\widowpenalty=500

%\usepackage[perpage]{footmisc} %Range of footnote options


% *****************************************************************************
% *************************** Bibliography  and References ********************

%\usepackage{cleveref} %Referencing without need to explicitly state fig /table

% Add `custombib' in the document class option to use this section
%\ifuseCustomBib
%   \RequirePackage[square, sort, numbers, authoryear]{natbib} % CustomBib
%\fi

% If you would like to use biblatex for your reference management, as opposed to 
%the default `natbibpackage` pass the option `custombib` in the document class. 
%Comment out the previous line to make sure you don't load the natbib package. 
%Uncomment the following lines and specify the location of references.bib file
%\usepackage[utf8]{inputenc}
\DeclareUnicodeCharacter{25B}{$\epsilon$}
\DeclareUnicodeCharacter{3F5}{$\epsilon$}
\DeclareUnicodeCharacter{2212}{-}
\DeclareUnicodeCharacter{301}{'}

\RequirePackage[backend=bibtex, style=numeric-comp, citestyle=numeric, sorting=none]{biblatex}
%\RequirePackage[backend=bibtex,style=chem-acs]{biblatex}
%\RequirePackage[backend=bibtex,style=phys]{biblatex}
%\bibliography{Mendeley/DenseGasDispersion.bib,Mendeley/Book.bib,Mendeley/ABL.bib,Mendeley/foam.bib} %Location of references.bib only for biblatex
\bibliography{library.bib}

% Colors hyperlinks in blue - change to black if annoying
\hypersetup{urlcolor=blue, colorlinks=true} 

% changes the default name `Bibliography` -> `References'
\renewcommand{\bibname}{References}


% ******************************************************************************
% ************************* User Defined Commands ******************************
% ******************************************************************************

% *********** To change the name of Table of Contents / LOF and LOT ************

%\renewcommand{\contentsname}{My Table of Contents}
%\renewcommand{\listfigurename}{My List of Figures}
%\renewcommand{\listtablename}{My List of Tables}


% ********************** TOC depth and numbering depth *************************
\setcounter{secnumdepth}{2}
\setcounter{tocdepth}{2}


% ******************************* Nomenclature *********************************
% To change the name of the Nomenclature section, uncomment the following line

%\renewcommand{\nomname}{Symbols}

% Improved version
\usepackage{nomencl,etoolbox,ragged2e,siunitx,mathtools}
\makenomenclature
%\DeclarePairedDelimiter{\abs}{\lvert}{\rvert}
%2nd edition
\iftrue
% Command for units column
\newcommand{\UnitsCol}[1]{\hfill\parbox[t]{4em}{#1}\ignorespaces}
\newcommand{\FormularCol}[1]{\hfill\parbox[t]{4em}{#1}\ignorespaces}

% Command for subtitle
\newcommand{\nomsubtitle}[1]{\item[\large\bfseries #1]}
\renewcommand\nomgroup[1]{\def\nomtemp{\csname nomstart#1\endcsname}\nomtemp}

\newcommand{\nomstartA}{\nomsubtitle{Acronyms}%
	\item[\bfseries Symbol]%	
	\textbf{Description}}
\newcommand{\nomstartS}{\nomsubtitle{Subscripts}%
	\item[\bfseries Symbol]%
	\textbf{Description}}
\newcommand{\nomstartR}{\nomsubtitle{Roman Symbols}%
	\item[\bfseries Symbol]%	
	\textbf{Description}%
	\UnitsCol{\textbf{Units}}}
\newcommand{\nomstartG}{\nomsubtitle{Greek Symbols}%
	\item[\bfseries Symbol]%
	\textbf{Description}%
	\UnitsCol{\textbf{Units}}}
\newcommand{\nomstartD}{\nomsubtitle{Dimensionless Numbers/Quantities}%
	\item[\bfseries Symbol]%
	\textbf{Description}%
	\FormularCol{\textbf{Definition}}}

\renewcommand*{\nompreamble}{\markboth{\nomname}{\nomname}}

\newcommand{\nomdescr}[1]{\parbox[t]{10cm}{\RaggedRight #1}}

\newcommand{\nomtypeA}[3][]{\nomenclature[A#1]{#2}{\nomdescr{#3}}}
\newcommand{\nomtypeS}[3][]{\nomenclature[S#1]{#2}{\nomdescr{#3}}}
\newcommand{\nomtypeR}[4][]{\nomenclature[R#1]{#2}%
	{\nomdescr{#3}\UnitsCol{#4}}}
\newcommand{\nomtypeG}[4][]{\nomenclature[G#1]{#2}%
	{\nomdescr{#3}\UnitsCol{#4}}}
\newcommand{\nomtypeD}[4][]{\nomenclature[D#1]{#2}%
	{\nomdescr{#3}\FormularCol{#4}}}
% Including definition
%\newcommand{\nomtypeD}[4][]{\nomenclature[D#1]{#2}
%	{\nomdescr{#3}\DefinitionCol{#4}}}
\fi


% ********************************* Appendix ***********************************
% The default value of both \appendixtocname and \appendixpagename is `Appendices'. These names can all be changed via:

%\renewcommand{\appendixtocname}{List of appendices}
%\renewcommand{\appendixname}{Appndx}


% *********************** Configure Draft Mode **********************************
% Uncomment to disable figures in `draft'
%\setkeys{Gin}{draft=true}  % set draft to false to enable figures in `draft'

% These options are active only during the draft mode
% Default text is "Draft"
%\SetDraftText{DRAFT}

% Default Watermark location is top. Location (top/bottom)
%\SetDraftWMPosition{bottom}

% Draft Version - default is v1.0
%\SetDraftVersion{v1.1}

% Draft Text grayscale value (should be between 0-black and 1-white)
% Default value is 0.75
%\SetDraftGrayScale{0.8}


% ******************************** Todo Notes **********************************
%% Uncomment the following lines to have todonotes.

%\ifsetDraft
%	\usepackage[colorinlistoftodos]{todonotes}
%	\newcommand{\mynote}[1]{\todo[author=kks32,size=\small,inline,color=green!40]{#1}}
%\else
%	\newcommand{\mynote}[1]{}
%	\newcommand{\listoftodos}{}
%\fi

% Example todo: \mynote{Hey! I have a note}
